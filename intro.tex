\section{Introduction}

\subsection{The First Handout: Course Description and Problem Statements}
\begin{center}
Fall, 2015\\
Mondays and Wednesdays, 1:10–2:25, Prof. K. Ross\\
688 Mudd Building
\end{center}

\subsection*{The Course}

This course is like no other course you've taken.  There are no exams.
There are no homeworks.  There are 4 projects, but all of the projects
are open-ended: there are no ``correct'' project answers.  The idea is
to develop problem solving skills using techniques that you have learned
during your CS training to solve the
project problems.  Lectures will be relatively informal, with most
of the class interaction being cooperative problem solving and discussion
rather than a fixed set of lectures with questions.

Some of the supplied problems may be too tough to make much progress on
in the 3 weeks you have.  If so, you should try to find special cases
of the problem that can be solved, or to simplify the problem in
such a way that it still remains interesting, but may be solved.

Conversely, some of the problems may be too easy.  In that case you
should try generalizing the problem.  To what other kinds of problem
can your solution also be applied?

For those of you familiar with \htmladdnormallink{Bloom's
  taxonomy}{http://en.wikipedia.org/wiki/Bloom's_taxonomy}, this class
emphasizes the upper levels (application, analysis, synthesis,
evaluation), and assumes the lower levels (knowledge, comprehension)
from your previous CS background.

All projects will be done in groups of three or four.  There are four
separate projects, and groups will change from project to project.
{\em You cannot be in the same group as another person more than
twice.}  We will hopefully be able to mix up the groups sufficiently
so that no pair of students is in the same group even twice.

A large proportion of your grade will come from class participation.
{\em Class attendance is mandatory.}  Do not sign up for this class if
you think you will need to miss classes.  An attendance record will be
kept by the TA.

This is the kind of course in which you can invest a little or a lot,
and for which your enjoyment and fulfilment will come in proportion
to the amount of energy you put into the course.

\subsection*{Prerequisites}
W3137 or W3134 (Data Structures),
and W3827 
(Fundamentals of Computer Systems).  
%W3824 (Computer Organization) can substitute for W3827.
Recommended: W4701 (Artificial Intelligence).
{\bf Permission of the instructor is required for all students.}
Senior
CS or Computer Engineering majors and CS graduate students only.
(Advanced juniors whose CS preparation is comparable to that
of seniors will also be considered.)
Enrollment will be limited to 30 students.

Permission to register for the class will be given by the instructor
{\em after the first class meeting}
according to the policies in the section ``Registering for the Course''
below.

\subsection*{Office Hours}
\begin{description}
\item[Instructor:] Prof.\ Ken Ross, 510 Computer Science Building,
212-939-7058, {\tt kar@cs.columbia.edu}

Office hours: Mondays 2:30--3:30pm, Thursdays 2--3pm.

\item[Teaching Assistant:] Orestis Polychroniou, 725 CEPSR, {\tt orestis@cs.columbia.edu}

Office hours: Tuesdays 11am--1pm.

\item[Teaching Assistant:] Georgios Koloventzos, {\tt gkoloven@cs.columbia.edu}

Office hours: Fridays 11am--1pm in the TA Room (1st floor Mudd).

\end{description}

\subsection*{Grading}

Projects will be graded by the instructor.  However, there is no
predefined grading scheme.  A grade will be given based on the
approach used to solving the given problem, including
\begin{itemize}
\item the novelty of the approach
\item the thoroughness of the project report
\item the clarity of the report presentation
\item the quality of the code
\item the correctness and generality of the proposed solution
\item the efficiency of the proposed solution
\end{itemize}

Several of the projects may have competitions to see whose program
``wins'' against other programs.  The outcome of these competitions is
not a factor in the project grades.  The competitions are intended to
stimulate the development of new ideas, and it is the ideas that are
important.  The competition rankings are for fun and pride only.

There is no restriction on communication between different groups.
However, if a certain idea X is proposed by group A, and group B
thinks the idea is so cool that they want to use it too, then group
B can do so {\em as long as group B explicitly cites group A's
contribution in their report.}  So, for example, suppose that
during the second class discussing project 1, a student Jane Smith reports
to the class
that she has discovered that a certain algorithm in the literature
applies to the problem.  Then other groups are still free to use that
algorithm too, but in their report they must include an ``acknowledgements''
section with text something like
\begin{quotation}
We acknowledge the contribution of Jane Smith who observed that the
17-starving-philosophers-named-Josephus algorithm applies to this
problem.
\end{quotation}
Any information supplied by the instructor or
teaching assistant need not be
acknowledged in this way.  Published sources and
Web sources should be formally cited using a bibliography.

If a group develops a tool that is potentially useful to other groups,
then you are encouraged to share it (and it's use should be
acknowledged).  However, this sharing should be limited to tools that
aid the development of solutions (eg., visualizers, compilers, etc.)
and not the solutions themselves.  Similarly, external data that might
be useful should be shared.  For example, if we had a map-drawing
project, then if somebody downloaded from the Web some datasets describing
real-world maps that could be used to test project code, that data
should be shared and used with acknowledgement.  (For the purposes
of this course, the use of other people's contributions without
acknowledgement is considered cheating.)

Outside people may be consulted for general references, but shouldn't
be asked to contribute to solving the problems themselves.  This is a
course whose primary purpose is to give you the opportunity to
tackle challenging problems yourselves.
If you consult outside people, they should
also be acknowledged.

While groups are permitted to communicate with each other, they are
not encouraged to work so closely together that they really could be
described as an 8-person group.  Groups must submit their own reports,
and the reports should be written independently of other groups.

A report must also include a section titled ``Summary of
Contributions'' that describes (in one or two sentences per person)
what each member of the team contributed to the project.

Reports should also contain (where applicable) a paragraph documenting
the use of code written by you that is used in a substantive way in
the submissions from other groups.  After the final code deliverable,
you have the opportunity to browse through the other groups' submitted
code to identify such use.  Remember that this kind of use is
encouraged, and hopefully in most cases the other groups will have
told you directly that they are working with your code.  The purpose
of including this paragraph is to make sure that the instructor can
see where groups have had influence on other groups.

Reports should be submitted electronically, as well as in hardcopy on the
day of the group presentation.  The electronic reports will be made
readable by all members of the class after the submission deadline.

Here are the grading proportions:

\begin{tabular}{ll}
Project Reports and Presentations & 60\% \\
Class Participation & 40\%
\end{tabular}

All members of a group will, by default, get the same grade for their
project report.  In rare circumstances, the professor will adjust the
grade of an individual member of a project team if there is clear
evidence that this member did not make a genuine effort to work on
the project.

In addition, you are required to attend all classes.  You can miss
two classes without penalty.  However, the third class missed will
result in a drop in the {\em final grade} of one fractional grade
(eg., B+ to B, or A to A-).  If you miss four classes, you will
drop a full grade (eg., A- to B-).  If you miss five or more
classes, you will fail the course.  Arriving in class more than
15 minutes late counts as missing the class.  (In the event of
a medical or other emergency, consult the instructor.)


\subsection*{Policies}

There will be no extensions for projects.  Hand in what you have on the due
date.

An electronic newsgroup will be set up for all course-related
announcements.  Check there regularly for
any course related announcements.  Questions or comments of a general
nature that would be of interest to the whole class should be posted
there.  The instructor and the TA will be reading the newsgroup
regularly.

The web site for the course is at
\htmladdnormallink{http://www.cs.columbia.edu/\~{}kar/4444f15}{http://www.cs.columbia.edu/~kar/4444f15}.
The TA will be taking notes of our discussions during class, and those
notes will be posted on the TA page.  Links to
other relevant resources will be
put on the web page or on courseworks.

You can use any computing platforms you want as long as your programs
conform to the interface specifications described within each project.
You will probably need \htmladdnormallink{CS
accounts}{https://www.cs.columbia.edu/~crf/accounts/cs.html} for this
class.

The number of times that a single student may contribute during one class
is limited to three, in order to allow for as many participants as possible.

\subsection*{Tentative Timetable}

\begin{description}
\item[Sep 9] Class introduction and overview.
Brief discussion of all 4 project problems.  Students submit questionnaires.
\item[Sep 11] Announcement of selected students on the class web page.
HW for Sep 14: Read thoroughly the description of project 1.
\item[Sep 14, 16, 21, 23, 28] 
Formation of teams for project 1.
Project 1 discussion.
\item[Sep 30] Project 1 final code submission deadline.
Formation of teams for project 2.
Project 2 discussion.
\item[Oct 5] Project 1 reports and presentations.
\item[Oct 7, 12, 14, 19] Project 2 discussion.
\item[Oct 21] Project 2 final code submission deadline.
Formation of teams for project 3.
Project 3 discussion.
\item[Oct 26] Project 2 reports and presentations.
\item[Oct 28, Nov 4, 9, 11] Project 3 discussion.
\item[Nov 13] Project 3 final code submission deadline. 
\item[Nov 16] Project 3 reports and presentations. Formation of teams for project 4.
\item[Nov 18, 23, 25, 30, Dec 2, 7] Project 4 discussion.
\item[Dec 9] Project 4 final code submission deadline. Course wrapup.
\item[Dec 14] Project 4 reports and presentations. Peer-review submission. 
\end{description}

\subsection{Participants}

The names of participating students are listed below.  If you are listed but do not intend to
take the class, please let the instructor know ASAP by email, so that
he can determine whether to accept students from the standby-list.
(Members of the standby-list have been notified by the instructor via email.)

\begin{tabular}{ll}
Name & Email (@columbia.edu unless specified) \\ \hline
Sania Arif & sa3311 \\
Alice Chang & avc2120 \\
Manyi Chen & mc3962 \\
Amar Dhingra & asd2157 \\
Samarth Dhingra & sd2900 \\
Preetam Dutta & preetam@cs \\
Vinay Gaba & vinay.gaba \\
Rhea Goel & rg2936 \\
Jing Guo & jg3527 \\
Xingzhou He & x.he \\
Nam Hoang & nnh2110 \\
Konstantin Itskov & koi2104 \\
Naman Jain & nj2303 \\
Cathy Jin & ckj2111 \\
Tingting Li & tl2617 \\
Sean Liu & sl3497 \\
Parthiban Loganathan & pl2487 \\
Diana Liskovich & dl2956 \\
Seth Mishan & sm3890 \\
Ananya Poddar & ap3317 \\
Dhruv Purushottam & dp2631 \\
Artur Renault & artur.renault \\
Sagar Sarda & ss4355 \\
Kevin Shi & kshi@cs \\
Vishal Vyas & vnv2102 \\
Liuyang Wang & lw2635 \\
Robert Ying & ry2242 \\
\end{tabular}
