\section{Introduction}

\subsection{The First Handout: Course Description and Problem Statements}
\begin{center}
Fall, 2015\\
Mondays and Wednesdays, 1:10–2:25, Prof. K. Ross\\
688 Mudd Building
\end{center}

\subsection*{The Course}

This course is like no other course you've taken.  There are no exams.
There are no homeworks.  There are 4 projects, but all of the projects
are open-ended: there are no ``correct'' project answers.  The idea is
to develop problem solving skills using techniques that you have learned
during your CS training to solve the
project problems.  Lectures will be relatively informal, with most
of the class interaction being cooperative problem solving and discussion
rather than a fixed set of lectures with questions.

Some of the supplied problems may be too tough to make much progress on
in the 3 weeks you have.  If so, you should try to find special cases
of the problem that can be solved, or to simplify the problem in
such a way that it still remains interesting, but may be solved.

Conversely, some of the problems may be too easy.  In that case you
should try generalizing the problem.  To what other kinds of problem
can your solution also be applied?

For those of you familiar with \htmladdnormallink{Bloom's
  taxonomy}{http://en.wikipedia.org/wiki/Bloom's_taxonomy}, this class
emphasizes the upper levels (application, analysis, synthesis,
evaluation), and assumes the lower levels (knowledge, comprehension)
from your previous CS background.

All projects will be done in groups of three or four.  There are four
separate projects, and groups will change from project to project.
{\em You cannot be in the same group as another person more than
twice.}  We will hopefully be able to mix up the groups sufficiently
so that no pair of students is in the same group even twice.

A large proportion of your grade will come from class participation.
{\em Class attendance is mandatory.}  Do not sign up for this class if
you think you will need to miss classes.  An attendance record will be
kept by the TA.

This is the kind of course in which you can invest a little or a lot,
and for which your enjoyment and fulfilment will come in proportion
to the amount of energy you put into the course.

\subsection*{Prerequisites}
W3137 or W3134 (Data Structures),
and W3827 
(Fundamentals of Computer Systems).  
%W3824 (Computer Organization) can substitute for W3827.
Recommended: W4701 (Artificial Intelligence).
{\bf Permission of the instructor is required for all students.}
Senior
CS or Computer Engineering majors and CS graduate students only.
(Advanced juniors whose CS preparation is comparable to that
of seniors will also be considered.)
Enrollment will be limited to 30 students.

Permission to register for the class will be given by the instructor
{\em after the first class meeting}
according to the policies in the section ``Registering for the Course''
below.

\subsection*{Office Hours}
\begin{description}
\item[Instructor:] Prof.\ Ken Ross, 510 Computer Science Building,
212-939-7058, {\tt kar@cs.columbia.edu}

Office hours: Mondays 2:30--3:30pm, Thursdays 2--3pm.

\item[Teaching Assistant:] Orestis Polychroniou, 725 CEPSR, {\tt orestis@cs.columbia.edu}

Office hours: Tuesdays 11am--1pm.

\item[Teaching Assistant:] Georgios Koloventzos, {\tt gkoloven@cs.columbia.edu}

Office hours: Fridays 11am--1pm in the TA Room (1st floor Mudd).

\end{description}

\subsection*{Grading}

Projects will be graded by the instructor.  However, there is no
predefined grading scheme.  A grade will be given based on the
approach used to solving the given problem, including
\begin{itemize}
\item the novelty of the approach
\item the thoroughness of the project report
\item the clarity of the report presentation
\item the quality of the code
\item the correctness and generality of the proposed solution
\item the efficiency of the proposed solution
\end{itemize}

Several of the projects may have competitions to see whose program
``wins'' against other programs.  The outcome of these competitions is
not a factor in the project grades.  The competitions are intended to
stimulate the development of new ideas, and it is the ideas that are
important.  The competition rankings are for fun and pride only.

There is no restriction on communication between different groups.
However, if a certain idea X is proposed by group A, and group B
thinks the idea is so cool that they want to use it too, then group
B can do so {\em as long as group B explicitly cites group A's
contribution in their report.}  So, for example, suppose that
during the second class discussing project 1, a student Jane Smith reports
to the class
that she has discovered that a certain algorithm in the literature
applies to the problem.  Then other groups are still free to use that
algorithm too, but in their report they must include an ``acknowledgements''
section with text something like
\begin{quotation}
We acknowledge the contribution of Jane Smith who observed that the
17-starving-philosophers-Named-Josephus algorithm applies to this
problem.
\end{quotation}
Any information supplied by the instructor or
teaching assistant need not be
acknowledged in this way.  Published sources and
Web sources should be formally cited using a bibliography.

If a group develops a tool that is potentially useful to other groups,
then you are encouraged to share it (and it's use should be
acknowledged).  However, this sharing should be limited to tools that
aid the development of solutions (eg., visualizers, compilers, etc.)
and not the solutions themselves.  Similarly, external data that might
be useful should be shared.  For example, if we had a map-drawing
project, then if somebody downloaded from the Web some datasets describing
real-world maps that could be used to test project code, that data
should be shared and used with acknowledgement.  (For the purposes
of this course, the use of other people's contributions without
acknowledgement is considered cheating.)

Outside people may be consulted for general references, but shouldn't
be asked to contribute to solving the problems themselves.  This is a
course whose primary purpose is to give you the opportunity to
tackle challenging problems yourselves.
If you consult outside people, they should
also be acknowledged.

While groups are permitted to communicate with each other, they are
not encouraged to work so closely together that they really could be
described as an 8-person group.  Groups must submit their own reports,
and the reports should be written independently of other groups.

A report must also include a section titled ``Summary of
Contributions'' that describes (in one or two sentences per person)
what each member of the team contributed to the project.

Reports should also contain (where applicable) a paragraph documenting
the use of code written by you that is used in a substantive way in
the submissions from other groups.  After the final code deliverable,
you have the opportunity to browse through the other groups' submitted
code to identify such use.  Remember that this kind of use is
encouraged, and hopefully in most cases the other groups will have
told you directly that they are working with your code.  The purpose
of including this paragraph is to make sure that the instructor can
see where groups have had influence on other groups.

Reports should be submitted electronically, as well as in hardcopy on the
day of the group presentation.  The electronic reports will be made
readable by all members of the class after the submission deadline.

Here are the grading proportions:

\begin{tabular}{ll}
Project Reports and Presentations & 60\% \\
Class Participation & 40\%
\end{tabular}

All members of a group will, by default, get the same grade for their
project report.  In rare circumstances, the professor will adjust the
grade of an individual member of a project team if there is clear
evidence that this member did not make a genuine effort to work on
the project.

In addition, you are required to attend all classes.  You can miss
two classes without penalty.  However, the third class missed will
result in a drop in the {\em final grade} of one fractional grade
(eg., B+ to B, or A to A-).  If you miss four classes, you will
drop a full grade (eg., A- to B-).  If you miss five or more
classes, you will fail the course.  Arriving in class more than
15 minutes late counts as missing the class.  (In the event of
a medical or other emergency, consult the instructor.)


\subsection*{Policies}

There will be no extensions for projects.  Hand in what you have on the due
date.

An electronic newsgroup will be set up for all course-related
announcements.  Check there regularly for
any course related announcements.  Questions or comments of a general
nature that would be of interest to the whole class should be posted
there.  The instructor and the TA will be reading the newsgroup
regularly.

The web site for the course is at
\htmladdnormallink{http://www.cs.columbia.edu/\~{}kar/4444f15}{http://www.cs.columbia.edu/~kar/4444f15}.
The TA will be taking notes of our discussions during class, and those
notes will be posted on the TA page.  Links to
other relevant resources will be
put on the web page or on courseworks.

You can use any computing platforms you want as long as your programs
conform to the interface specifications described within each project.
You will probably need \htmladdnormallink{CS
accounts}{https://www.cs.columbia.edu/~crf/accounts/cs.html} for this
class.

The number of times that a single student may contribute during one class
is limited to three, in order to allow for as many participants as possible.

\subsection*{Tentative Timetable}

\begin{description}
\item[Sep 9] Class introduction and overview.
Brief discussion of all 4 project problems.  Students submit questionnaires.
\item[Sep 11] Announcement of selected students on the class web page.
HW for Sep 14: Read thoroughly the description of project 1.
\item[Sep 14, 16, 21, 23, 28] 
Formation of teams for project 1.
Project 1 discussion.
\item[Sep 30] Project 1 final code submission deadline.
Formation of teams for project 2.
Project 2 discussion.
\item[Oct 5] Project 1 reports and presentations.
\item[Oct 7, 12, 14, 19] Project 2 discussion.
\item[Oct 21] Project 2 final code submission deadline.
Formation of teams for project 3.
Project 3 discussion.
\item[Oct 26] Project 2 reports and presentations.
\item[Oct 28, Nov 4, 9, 11] Project 3 discussion.
\item[Nov 13] Project 3 final code submission deadline. 
\item[Nov 16] Project 3 reports and presentations. Formation of teams for project 4.
\item[Nov 18, 23, 25, 30, Dec 2, 7] Project 4 discussion.
\item[Dec 9] Project 4 final code submission deadline. Course wrapup.
\item[Dec 14] Project 4 reports and presentations. Peer-review submission. 
\end{description}

\subsubsection{Project 1: Parallel Pied Piper Plaza}
Last year we had a project called ``Parallel Pied Pipers'' whose
description can be found
\htmladdnormallink{here}{http://www.cs.columbia.edu/~kar/4444f14/node16.html}.
(You should read last year's description before continuing.)
This year we're going to use the same basic idea, but now it's a
multi-player game!  The shape of the field that initially contains the
rats is a square of size $F$ meters.  In the middle of each of the
four sides of the field is a 2m wide gate.  Each of the four teams of
pipers tries to get rats though their own gate (they are getting paid
per captured rat).  Once a rat goes through the gate it exits the
game.  There is a small holding area outside each gate, of size $F$
meters by 10 meters, where a piper can move if he wishes.  All pipers start
spaced out within the 2m gate.

When pipers are away from other pipers, or are near other pipers from
their own team, the piper/rat behavior is the same as last year's
project.  However, because different piper teams play different music
genres, rats are confused by the dissonance of different tunes.  In
particular, consider a rat that can hear the music of multiple pipers
from different teams because they are all within 10m of the rat and
playing music.  If there is a single dominant tune (e.g., one of the
teams has more nearby pipers than any single other team) then the rat
will be attracted to the closest of the pipers from that team.
Otherwise, the rat will behave as if there is no sound, e.g., the rat
will not be attracted to any of the pipers.

Each of the four teams has $d$ pipers.  Each piper is executing an
instance of the same code, but instantiated with a unique piper-id
parameter (between 1 and $d$).  Pipers have complete access to all
positional information, including the positions and music-playing status of
the other teams' pipers, but pipers do not explicitly communicate with other
pipers.

The simulator will detect when all rats have been eliminated, and you will be ranked based
on the number of rats that exited though your gate.  There will be a timeout to
handle cases where some rats simply can't be convinced
to go though any gate.  We will run some tournaments at the end of class
with various values for the parameters $F$, $S$ (the number of rats), and $d$.

You will have access to the players and reports from last year's
teams.  You can choose to use code written last year if you wish,
making sure to make suitable acknowledgements.  Alternatively, you can
write your own code from scratch if you prefer.

Some ideas to think about:
\begin{itemize}
\item What kinds of coordinated teamwork (if any) are needed in this multiplayer version of the game?
\item What is the impact of the dissonance of music from multiple nearby pipers?
\end{itemize}

\subsubsection{Project 2: Planet Builder}

Asteroids of equal size orbit the sun in initially circular orbits at
various radii.  There is a lower and upper radius bound, so that
the asteroids are in a formation like the asteroid belt between
Mars and Jupiter.  You get to
push the asteroids a bit to change their trajectory.  When asteroids
collide there is conservation of momentum and the asteroids remain
joined (and larger).  Energy is not conserved: the newly formed
asteroid heats up!  Your goal is to create a single planet made up of at
least half the initial asteroid mass within a fixed time $T$.  Each
time you exert a push, the energy required for that push is
accumulated into your score; you want to minimize the total energy
used. (Using a lot of energy to form a planet quickly scores worse
than forming the planet more slowly with less energy, as long as the
planet forms within $T$ time units.)

Asteroids are influenced in their trajectory by the sun, whose gravity
is strong.  The gravitational attraction of other asteroids is assumed
to be negligible and is ignored by the simulator.  All interactions
(including your pushing of asteroids) happen in the plane of rotation
of the solar system.  A push perpendicular to the movement of the
asteroid will change its orbit, but sill retain the circularity of the
orbit; a non-perpendicular push (or collision) may lead to an
elliptical orbit.

For a cool related game concept, see \htmladdnormallink{Osmos}{http://www.osmos-game.com/}.
\subsubsection{Project 3: Cookie Cutter II}

In 2003, I gave a \htmladdnormallink{project}{http://www.cs.columbia.edu/~kar/4444f03/node16.html}
involving cutting cookies from a piece of dough.  This year we're doing something different: competitive
cookie cutting!  This is a game for two players, played on a 50x50 grid of dough.  Each player designs three
cookie cutter shapes (details below) and tries to cut out as much of the dough as possible before the
dough runs out.  The score is the total area of dough that your player manages to capture.  Your primary
goal is to capture more dough than your opponent (as opposed to maximizing the amount of dough), so
defensive strategies that block your opponent while blocking yourself to a lesser degree are worth
considering.

You get to make three cutters of size 11, 8, and 5 units.  These
cutters must be a connected orthogonal set of unit squares.  So the 5-unit
shape must be one of the 12 possible
\htmladdnormallink{pentominos}{http://en.wikipedia.org/wiki/Pentomino},
for example. (Actually, since we don't consider reflective symmetries, there
are more than 12 choices: how many are there?)  Each player creates their 11-unit shape first and
submits it to the simulator.  The simulator shares this information
with both players, who then select their 8-unit shape.  After the
simulator informs both players of the 8-unit selection, players submit
their 5-unit shapes.  So players get to design the smaller shapes in
light of the known choices of both players for the larger shapes.

One final rule about selecting shapes.  The simulator will not allow players to select
the same shape (taking the four rotational symmetries into account).  If players
happen to select the same shape, both players' choices are rejected and players have to
select a new shape, different from all previous choices.  If the simulator has to reject
players' shapes 5 times in a row, then the simulator will randomly assign one of the five
choices to each player (a different shape to each).

The game is played by placing a cookie-cutter of your choice anywhere
on the board in one of the four orthogonal orientations.  You may only
select a location if there is dough remaining within each cell
delimited by the cookie cutter.  Note that the cookie-cutter cannot include partial dough
cells: the perimeter defined by the cookie-cutter must lie on the ``grid lines'' of the dough grid.
At the end of the turn, your score is
incremented by the size of the cutter used, and the corresponding
dough is removed from the grid.  Players take turns, but to balance the
game the first player must use the 5-unit cutter on the first turn.

Some things to think about:
\begin{itemize}
\item What are the benefits and risks of regular shapes compared with irregular shapes?
\item Near the beginning of the game, how should you space out your choices?
\item How does knowing your opponent's larger shapes help you select your smaller shapes?
\end{itemize}
\subsubsection{Project 4: Work the Room}

You're planning to attend a big party.  Some of your friends will be
there, but many people will be strangers.  Among the strangers is your
unknown soulmate.  Your goal is to have beneficial conversations with
as many people as you can for three simulated hours. Friends are
interesting to talk to, strangers less so, but your soulmate is
particularly interesting.  Also, you need to move around in order to
find people and chat with them.  The party takes place in a large
square room measuring 20 meters on each side.

Each turn of the simulator corresponds to a 6-second chunk of
real-world time. During any turn your player may attempt to do one of the following:
\begin{itemize}
\item Move to a new position within the room anywhere within 6 meters of your current position. Any conversation
you were previously engaged in is terminated: you have to be stationary to chat.
\item Initiate a conversation with somebody who is standing at a distance between 0.5
meters and 2 meters of you. In this case you do not move.
You can only initiate a conversation with somebody on the next turn
if they and you are both not currently engaged in a conversation with
somebody else. (So there has to be at least one turn between conversations.)
\item Continue a conversation that you were participating in during the previous turn. In this case you also do not move.
\end{itemize}

All players submit their orders together, and the outcome is resolved by the simulator using the following rules in sequence:
\begin{enumerate}
\item If two players were previously chatting, and both wish to continue the conversation, then
the conversation continues.
\item If two players, who were not previously chatting to anybody, simultaneously try to initiate a conversation
with each other, then a conversation between them is begun.
\item For each remaining player who tries to initiate a conversation, figure out how close each is to
their conversation target, and do the following in increasing distance order (with ties broken randomly):
\begin{enumerate}
\item If the target is not already engaged in conversation on the current turn, then begin a new conversation
between the two players.  The target cannot refuse (you have to be polite!) but can terminate the conversation on the
next turn.
The target's action for the current turn (whatever the action was) is cancelled and the target stays put.
\item If the target is already engaged in conversation on the current turn, then the player stays put and does
not begin a conversation.
\end{enumerate}
\item Players who try to continue a conversation with a partner who chooses to move away
just stay put, and the conversation is ended.
\item Any remaining players must be attempting to move.  All of those moves succeed.
This includes players who terminate a conversation by moving:
no other player would have been allowed to start a new conversation with them.
\end{enumerate}
Illegal moves (e.g., moving more than 6 meters in a turn, trying to start a conversation with somebody too close or
too far away, etc.) will be treated by the simulator as a ``stay put'' directive.

The purpose of a conversation is to accumulate
gossip/wisdom/life-lessons, that come conveniently packaged in
6-second units.  The wisdom that players can offer each other is
captured in a two-dimensional array $W$.  There are $N$ players in the
game, where $N$ is a parameter, so $W$ is an $N$ by $N$ array where
$W[i,j]$ represents the total wisdom that player $i$ has to offer
player $j$. ($W[i,i]$ is not meaningful.)  It is not always the case
that $W[i,j]=W[j,i]$. 

On a turn when players $i$ and $j$ are chatting, each has the
opportunity to collect a unit of wisdom from their partner.  However,
they will only obtain this wisdom if there is no other ``interfering''
player closer to them than their conversation partner (too hard to
hear otherwise!). It therefore pays to try to start your conversations
away from other players.  Note that it is possible for one
partner in the conversation to gain wisdom, while the other does not
due to interference.  If $W[i,j]$ is nonzero and nobody is closer to
$j$ than $i$ then the score of player $j$ is incremented by one and
$W[i,j]$ is decremented by one. A symmetric outcome happens for player
$i$. Once $W[i,j]$ reaches zero, $j$ has no further incentive to chat
with $i$.

Players have special relationships with other players.  Each player
has $f$ friends, $s$ strangers, and one soulmate among the other
players, where $f+1+s = N$. Players know who their friends are, but do
not know who their soulmate is.  Friendship is symmetric.  If $i$ is a
friend of $j$ then $j$ is a friend of $i$ and $W[i,j]=W[j,i]=50$. In
other words, friends can chat productively for up to 5 minutes.

Soulmates are also symmetric, but with $W[i,j]=W[j,i]=200$.  If $i$
and $j$ are neither friends nor soulmates, then $W[i,j]$ is chosen at
random from $\{ 0, 10, 20 \}$. To balance the game, the simulator will
make sure that each player has the same number of $0$, $10$ and $20$
values among all strangers. For nonfriends, when player $i$ chats with
player $j$, player $j$ learns $W[i,j]$ and player $i$ learns
$W[j,i]$. Prior to that initial conversation, $W[i,j]$ is unknown.
Player $i$ is unaware of the values of $W[i,j]$ for strangers $j$.

Finally, your player can see only a radius of 6m around its current position.
The simulator will tell you the positions of all players within that radius.
(Your player is free to remember previous locations of players that have
gone outside the 6m radius if that
information would be helpful later.) If two players who are both within
6m are having a conversation, the simulator will make your player aware
of this conversation. (How could that information be helpful?)
The values of $N$, $f$, and $s$ are known to each player at the start of the game.

We'll run several tournaments at the end of class using various
parameter settings for $N$, $f$, and $s$. Large values of $N$ are
possible, in which case we'll use multiple instances of players from
each group.

\subsection{Participants}

The Names of participating students are listed below.  If you are listed but do not intend to
take the class, please let the instructor know ASAP by email, so that
he can determine whether to accept students from the standby-list.
(Members of the standby-list have been notified by the instructor via email.)

\begin{tabular}{ll}
Name & Email (@columbia.edu unless specified) \\ \hline
Sania Arif & sa3311 \\
Alice Chang & avc2120 \\
Manyi Chen & mc3962 \\
Amar Dhingra & asd2157 \\
SAmarth Dhingra & sd2900 \\
Preetam Dutta & Preetam@cs \\
Vinay Gaba & vinay.gaba \\
Rhea Goel & rg2936 \\
Jing Guo & jg3527 \\
Xingzhou He & x.he \\
Nam Hoang & nnh2110 \\
Konstantin Itskov & koi2104 \\
Naman Jain & nj2303 \\
Cathy Jin & ckj2111 \\
Tingting Li & tl2617 \\
Sean Liu & sl3497 \\
Parthiban Loganathan & pl2487 \\
Diana Liskovich & dl2956 \\
Seth Mishan & sm3890 \\
Ananya Poddar & ap3317 \\
Dhruv Purushottam & dp2631 \\
Artur Renault & Artur.renault \\
Sagar Sarda & ss4355 \\
Kevin Shi & kshi@cs \\
Vishal Vyas & vnv2102 \\
Liuyang Wang & lw2635 \\
Robert Ying & ry2242 \\
\end{tabular}

\subsection{Project Teams}

The following assignments were made during the course of the class. Groupings were random, 
except that people who had previously worked together were (as far as possible) not 
grouped together for subsequent projects.
\\
\\
Project 1
\begin{enumerate}
\item Kostantin - Parthiban - Sagar
\item Kevin - Preetam - Naman
\item Amar - Dhruv - Sean
\item Tingting - Ananya - Vishal - Rhea
\item Sania - Vinay - Seth
\item xingzhou - Robert - Alice - Nam
\item Manyi - Diana - SAmarth
\item Jing - Artur - Cathy - Liuyang
\end{enumerate}
Project 2
\begin{enumerate}
\item Preetam - Vinay - SAmarth
\item Sean - Derek - Parthiban
\item Robert - Cathy - Manyi
\item Dhruv - Artur - Alice
\item Amar - Sagar - Vishal
\item Konstantin - Tingting - Nam
\item Kevin - Ananya - Diana
\item Naman - Sania - Jing
\item Rhea - Seth - Liuyang
\end{enumerate}
Project 3
\begin{enumerate}
\item Konstantin - Preetam - Vishal
\item Sagar - Dhruv - Vinay
\item Derek - Kevin - Sania
\item Artur - Sean - Nam
\item Ananya - Amar - Jing
\item Parthiban - Robert - Diana
\item Naman - SAmarth - Liuyang
\item Cathy - Alice - Rhea
\item Manyi - Tingting - Seth
\end{enumerate}
Project 4
\begin{enumerate}
\item Alice - Sean - Vishal
\item Dhruv - Parthiban - Kevin
\item Artur - Amar - Preetam
\item Cathy - Nam - Diana
\item Tingting - Jing - Sagar
\item Manyi - Naman - Ananya
\item Liuyang - Konstantin - Vinay
\item Rhea - Robert - Sania
\item SAmarth - Derek - Seth
\end{enumerate}
