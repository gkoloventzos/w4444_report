\section{Cookie Cutter II}
\subsection{Wednesday October 21, 2015}
First some questions were asked from the TAs about the randomness of their planet builder.
In order not to run many experiments with same configurations we will do pseudo randomness with a specific seed in Random. 
Also the students said that using maps from today will not be fair as if it is picked the particular team will have an advantage.

Problem 3 Cookie cutter

Students were asking what happens when illegal move is detected.
The simulator throws exception.
Cathy asked which size and what cutter a player can use.
Except the first turn the user can choose whatever he wants.
Rhea asked if they can have holes in the cutters. Yes you can
Seth asked about how the cutters can be placed. - No mirror only spin around.
Kevin asked about what metrics should we care. Dough covered or win - Win
Preetam asked a strategy question about U shaped cutters but was get on hold.
Tingting wandered if they have any limitations on how many times they can use a cutter. No limit
Kevin wanted to know if they will be cpu limit. Probably yes.

Then we did the group assignment 

Strategy questions:

Artur said that choosing shape is important. Your pieces must give advantage to your pieces and blocking your opponent.
Tingting asked what we mean by helpful. It means that your 8 piece can be helpful to your 11 piece (creating a weird shape that only your pieces can fit)
Kevin suggested that the 8 piece cannot block the 11 piece.
Vinay’s opinion is to create pieces that can fit together
Namar said that you big pieces can create space for the small one to fit.
Amar thinks that there are 2 main strategies. One that ignores opponent movement until a point and one that competes from start
Dhruv believes that making gaps is the winning point as you can always fill it near the end.
Sagar states that the pieces should mend together.
Preetam suggests the usage of the 11 piece more as it is more greedy to get dough.
Sania states that if you figure out the opponent strategy you should start be adversarial.
Cathy strategy is to try to fill the board until a limit. There are many options in the beginning so go random at the beginning and then adversarial
Parthi believes that you should go for the edges at the beginning because at the end will be difficult to fit.
Jing is against parthi strategy because the corners can be easily blocked by one piece of the opponent.
Sean believes that blocking the 11 piece of the adversary is the winning strategy
Robert believes an L level cutter and with a cascading strategy can block the opponent.
Vishal point that an S shape will be even better.
Alice states that if one chooses L and the opponent S then you are in a worse position
Derek point of view is that if the opponent chooses L choose cube in the smaller piece
Nam states that cube can be attacked easily because of the shape.
Ananya believes that this is not one way analysis of the cube. Cube can ruin also L shape cutters
Rhea said that using irregular cutters can leave more dough unusable as no one regular piece will fit in those.
Preetam suggests that you should create dead space for you opponent.
Seth agrees that as you know what shapes he has you can create dead space he cannot cover
Lingyuam suggests that the cutter should be horizontal in order to subdivide better
Costa believes that you can always defeat a strategy of covering pieces by placing next to the opponent piece to block him
Tingting adds that there is a flexibility on where you can block as you go all directions
Manyi suggests that the 11 and 8 piece combined can create a hole that fits your 5 piece

Diana believes that you 5 piece should interfere with the hole of the 11 piece of your opponent
Nam said that the choice of the 11 piece is crucial as it it the one that covers most of the dough
If you can create a U shape you can interfere more.
Manyi suggests that the strategy can find what shapes you can create and attack them
Preetam believes that the 11 and 5 pieces must be constructive.
Vinay suggestion is to find the complementary piece of your opponent
Meaning that it should fit the 11 piece
Kevin suggests that having an irregular 11 piece can not stuck perfectly so a line piece is better.
\subsection{Wednesday October 28, 2015}
First some questions were asked from the TAs about the randomness of their planet builder.
In order not to run many experiments with same configurations we will do pseudo randomness with a specific seed in Random. 
Also the students said that using maps from today will not be fair as if it is picked the particular team will have an advantage.

Problem 3 Cookie cutter

Students were asking what happens when illegal move is detected.
The simulator throws exception.
Cathy asked which size and what cutter a player can use.
Except the first turn the user can choose whatever he wants.
Rhea asked if they can have holes in the cutters. Yes you can
Seth asked about how the cutters can be placed. - No mirror only spin around.
Kevin asked about what metrics should we care. Dough covered or win - Win
Preetam asked a strategy question about U shaped cutters but was get on hold.
Tingting wandered if they have any limitations on how many times they can use a cutter. No limit
Kevin wanted to know if they will be cpu limit. Probably yes.

Then we did the group assignment 

Strategy questions:

Artur said that choosing shape is important. Your pieces must give advantage to your pieces and blocking your opponent.
Tingting asked what we mean by helpful. It means that your 8 piece can be helpful to your 11 piece (creating a weird shape that only your pieces can fit)
Kevin suggested that the 8 piece cannot block the 11 piece.
Vinay’s opinion is to create pieces that can fit together
Namar said that you big pieces can create space for the small one to fit.
Amar thinks that there are 2 main strategies. One that ignores opponent movement until a point and one that competes from start
Dhruv believes that making gaps is the winning point as you can always fill it near the end.
Sagar states that the pieces should mend together.
Preetam suggests the usage of the 11 piece more as it is more greedy to get dough.
Sania states that if you figure out the opponent strategy you should start be adversarial.
Cathy strategy is to try to fill the board until a limit. There are many options in the beginning so go random at the beginning and then adversarial
Parthi believes that you should go for the edges at the beginning because at the end will be difficult to fit.
Jing is against parthi strategy because the corners can be easily blocked by one piece of the opponent.
Sean believes that blocking the 11 piece of the adversary is the winning strategy
Robert believes an L level cutter and with a cascading strategy can block the opponent.
Vishal point that an S shape will be even better.
Alice states that if one chooses L and the opponent S then you are in a worse position
Derek point of view is that if the opponent chooses L choose cube in the smaller piece
Nam states that cube can be attacked easily because of the shape.
Ananya believes that this is not one way analysis of the cube. Cube can ruin also L shape cutters
Rhea said that using irregular cutters can leave more dough unusable as no one regular piece will fit in those.
Preetam suggests that you should create dead space for you opponent.
Seth agrees that as you know what shapes he has you can create dead space he cannot cover
Lingyuam suggests that the cutter should be horizontal in order to subdivide better
Costa believes that you can always defeat a strategy of covering pieces by placing next to the opponent piece to block him
Tingting adds that there is a flexibility on where you can block as you go all directions
Manyi suggests that the 11 and 8 piece combined can create a hole that fits your 5 piece

Diana believes that you 5 piece should interfere with the hole of the 11 piece of your opponent
Nam said that the choice of the 11 piece is crucial as it it the one that covers most of the dough
If you can create a U shape you can interfere more.
Manyi suggests that the strategy can find what shapes you can create and attack them
Preetam believes that the 11 and 5 pieces must be constructive.
Vinay suggestion is to find the complementary piece of your opponent
Meaning that it should fit the 11 piece
Kevin suggests that having an irregular 11 piece can not stuck perfectly so a line piece is better.
\subsection{Wednesday November 4, 2015}
Group 1-4
They choose similar shapes and it was awkward.
The players did not change much from the previous time.
Robert asked if the teams had a backup strategy if a piece selection failed. Because most of them they did not in the previous.
Preetam replied that at least them they are going with exhaustive search in the pieces they want.

Group 1 strategy tries to cover the board starting from the server and go circular around the center.
Group 4 goes left to right from the top left corner.
As the groups did not interfere the professor asked if it si good or bad to not interfere.
Seth replied that the group 1 lost because they mend 11 pieces together and do not let space for the 5 piece.
Tingting pointed out that utilizing the edges are creating space only for them.
Lingyuan answering the professor’s question said that is good to interfere but you may miss some opportunities while being focused only to interfere. 
Alice realises that the outcome would have been close to tie if they would not have mend the 11 piece together.
Manyi believes that if the other team is not interfering then you should not either.
Kevin says that at the end platin more the 11 piece is the goal. As these will get you more dough and probably create space for the smaller pieces.
Amar points that if we choose the opponents smaller piece then it will either work for us or it will not get the 100% of the mending.
Derek points out that placing toy piece in the center is more vulnerable to an opponent piece.
Sagar believes that the group 4 is not easily interfered.  Meaning as they go for edges their 11 piece is obtaining 16 piece dough as the opponent cannot interfere with the space inside.
Continuing Sagar suggests that with a straight line near the edge you are going for 22 the maximum for such piece.
Robert disagrees that you cannot interfere. You can if you go 1 or 2 lines from the edge.
Arthur also suggested that an irregular shape can help interfere. 


2 vs 7
Both went for straight line. At the end they got opposite L pieces.
Sagar has demonstrated how is possible to create an L shape corner only for them.
Nam analyzed that with the straight line can break the opponent piece but as they also had similar strategy they have problem.
Samarth also said that their strategy is to count how many moves of your opponent you destroy by your move.
Amar added that you should count also how many of your moves you increase or slightly decrease.


5 vs 9
Group 5 are using diagonal pieces.
Amar suggests that this piece creates space only their piece can go.
Seth points that their strategy is to have 8 and 5 pieces to mend with how the 11 piece is used.
Rhea points out that if the opponent is adversarial the strategy of the 9th group would be in trouble.
Cathy thinks that as the 5 group is using diagonal they create space that can be used only by them.
Ananya said that the 5 diagonal piece can fit to many places that are created in random.
Vishal points out that group 9 strategy does not create a closed space only for them.
Liuyang  said that if you piece is interrupting many diagonals then you are interrupting the groups 5 strategy.


6 vs 3
Kevin states that their strategy is to interrupt the convex hull of the opponent 11 piece.
Amar points that if opponent randomizes their moves then stacking can help.
Sagar asked what happens if the straight line cannot be chosen. THe answer is that are going for almost straight.


6 vs 2
Group 2 is choosing straight lines to cover most of the board.
Sania stated that group 6 went well even everybody thought otherwise
Amar believes that the problem was the 8 piece as straight line which was not used as much.

A question was asked about what is board control.
Sagar answered that is to create holes only for you to use in the future.
Costa stated that board control should derive from your strategy.
Robert suggested that it will be nice to see 2 groups that are trying to control the board.


5 vs 6
Robert points out that their strategy also can understand if an area has space only for them to fit not to play  it but go and interfere somewhere else.



deliverable: near final strategy.
Important things that should be dealt with:
Seth said that they should play with other players and find out where they fail.
Cathy suggested that should spot weakness and either attack them if in opponent or fix them.
Ananya believes a selection of shapes should have a bit more thought
Jing believes that the interfering should be dynamic and not straight forward.
\subsection{Monday November 9, 2015}
Thursday midnight will be the deadline for the cookie cutter project


6 vs 8
Straight pieces vs irregular
Group 8 are trying to be destructive by placing their piece near the opponent.
Vinay pointed out that they played early their 8 piece in order to block potential
move of the opponent.
Dhruv believes that the destructive strategy was the error that made the group 8 to lose.
Rhea points out that the irregular pieces can win the convex hull strategy
Sagar opinion is that placing a piece near the opponent’s is not the best destructive strategy
Alice said that this strategy is only good for the stacking strategy


1 vs 7
Lingyuan addresses the strategy that they are using: every cell has a possible number of moves that can be participate. So make this number minimal for opponent.
Sagar added that picking the 5 piece as 3x3 convex hull left less moves with 2 pieces
Costa thought that leaving 2 pieces can mean that the 11 could play.
Ananya pointed out that they did not at the end because they have placed them correctly
Robert believed that the destructive strategy is easier than constructive.
Preetam said that the point is not to be destructive to yourself.
Vishal added that tiling is not aggressive.


2 vs 4
straight vs piece with hole.
Seth believes that they lost as they did not place the piece near to edges in order to block the straight piece.
Sania added that the placing of destructive piece was not done properly leaving enough space for the opponent piece.
Cathy points out that the 8 piece of g2 was not placed at all.
Tingting added that the g4 destroyed the inside of their 11 point on purpose.
Artur said that this was the best move at this point to eliminate moves from the opponent
Nam suggested that blocking straight line horizontally can have better effects.
Costa adds that it is not horizontal or vertical but how the opponent is playing.
Vinay added that it is essential for this pieces with holes to prevent stacking.


5 vs 3
Dhruv points out that g5 left too much space between their pieces for the 11 piece of g3 to be played.
Ananya answered that they are trying to spread out to control the board.

A question have arisen if spreading out is good for the board control?
Robert answered that spreading with one piece is not good as it leaves too much space for the opponent.
Alice asked if the neighbor function returns if out of bounds. Orestis replied that it is not.

9 vs 6 
Parthi believed that L pieces can be a counter piece for straight lines
Orestis’ opinion is that diagonal is better on that.
Sagar adds that blocking line piece is hard.

Robert suggested to see 2 line strategies.


3 vs 6

Kevin believed that the end will decide who will win.
Parthi added that at the end you can have deeper search.
Alice believed that it would be better to limit your space in the space with more free space and do there a deep search.
Sagar was negative to this because at that time you may not be able to turn the tide i you r favor.

\subsection{Wednesday November 11, 2015}
Today there were some suggestions of what game we should see instead of randomly choise.

Dhruv suggested a game between diagonal and line
Sean wanted a game with line vs line to see fallback
Robert asked to see a  time constraint game as they were not sure what it will happen.


2 vs 3 

both go for linear, 2 go for diagonal 3 to hockey stick

Sagar said that they space they left was due to not create enough space for the opponent’s piece.
Tingting asked if they have done an exhaustive search.
Sagar replied that they did not. But still they left movements that can only played by them for the end.
Lingyan added that theirs 5 piece blocked some moves of the opponents thus it was palyed a bit early.
Ananya believed that this strategy is good because you are leaving space that no one can use and blocks the opponent.
Artur suggested that this is not blocking because you block one move. They may have 3 or 4.
Dhruv suggested that using so early the 5 you will lose much dough.
Vinay suggested that you should play the 8 piece to lose less.


From the discussion found that many teams were trying to create antistrategies for specific teams.
Professor asked if it is better if you know with whom you are against.
Preetam said that is much better.
Robert added that it is possible to have better result knowing your opponent.
Sagar added that by knowing the players they identified which strategies can win and create counter strategy to beat them.
Seth believes that it is fair as all can do it.
Derek suggested they might change the strategy in the next few hours.
Costa added that they will do whatever to win.
Professor said that this not exactly how the course is. It is build for collaboration between teams
Vinay prefer to detect a specific piece and build a counter strategy for this then it is fair.
Seth also added that you only know what the last deliverable and not what in the final will be.


8 vs 4
irregular vs stacking
stacking win but too much space not used because the 5 piece was blocked out.


8 vs 1
Robert adds that if you choose a 11 diagonal piece you should choose the smaller to be similar.
Sagar said try to understand you opponent. You should choose your 8 piece to be constructive with you and destructive for your opponent.
Dhruv added that in destructive strategies you have more free dough in the end.

8 vs 2
group 2 has some space left for the 11 at the end


5 vs 7
no destructive strategy
Artur said that is risky letting your opponent do whatever he likes.


7 vs 6
straight line tvs corner.

Preetam believed that there will be many closes matches
Sania added players will change a lot so players that do player checking may lose.
Nam believes that straight lines be the winner strategy.
Vinay straight lines was always a bit better.

Professor asked if there is any advantage to start first.
Sagar said  that it is not an advantage as you use your 5 piece instead of 11 (6 pieces less)
Seth added  that the one who put his 11 piece first will impose his will.

How to put the 5 piece in first turn.
Costa proposed not in the corner but somewhere to be destructive.
Nam randomly will not help you but also may help the opponent.
Tingting believed that a strategy, destructive or constructive, will decide where to place it.
\subsection{Monday November 16, 2015}
As in the previous presentations where 9 teams presented, the time of each group
will be 6 minutes. George will give you a signal when one minute is left.

Group 9:
Group 9 gave an overview of most of the strategies. Also for them the selection
of shapes goes as: Lines -> hokey shape -> L shape families. They use multiple 
strategies over time. Like they are looking if they can counter a team that has
an 5 piece that will fit a gap in the 11 piece. Their main strategy is a defensive
one which they create queues of L shapes through the center of the board. That
helps them to take control of the board and has a secondary strategy (save-for-later)
as they are creating space that only their 11 piece can fit. They also explain
two other constructive strategies. A square named one, that creates squares but
leave too much space open. An aggressive one that create queues of 11 piece shapes
with the same drawback as the previous one. They did not do as well as other teams
in overall victories but the play well against constructive teams but had the 
opposite for the destructive ones. Also the counter piece works good.

Group 5:
The team started with alphabetic shapes but they understood that using diagonal
pieces had better results. This strategy is constructive for the team but destructive
for the rest. In choosing pieces they always start from straight line in order to
block their opponent. They created a cost function of 

$cost = distanceFromCenterOfMass * 
\frac{\sum_{pieces}{sizeOfPiece*numberOfOurMovesOfSizeDestroyed}}{\sum_{pieces}{sizeOfPiece*numberOfOpponentMovesOfSizeDestroyed}}$

In the corner case of $numberOfOpponentMovesOfSizeDestroyed = 0$ then the cost 
is an value of $Integer.MAX_VALUE$. So they choose the move with the smallest cost
value. If all of moves with the 11 piece have $Integer.MAX_VALUE$ cost then these
moves can only be filled by them. Then they turn to he 8 piece. If all moves are
$Integer.MAX_VALUE$ then all of the remaining moves cannot affect the opponent.
So do whichever one is more constructive. They won many times against groups 
1,2,4,8,9 and lost against the rest. Against group 3 they were devastated as the
average score difference was 186 points. 

Group 2
Their strategy is to lookahead in depth 2. For every empty square they calculated
a score. The scoring function is to  

Their backup strategy is to choose diagonals or hooks. Both strategies are not 
optimized and this is depicted to the results. They placed 5th overall with
catholic lose to group 9 who had won 16 games out of 80. But they had the 2nd
average score per game.

Group 8:
The team's choice is to use irregular shape. They are looking on 1 depth search
for best placement. They also using a purely destructive placement. They also 
try to counter the 11 piece of the opponent. The placement is done by calculating
the centroid and place their piece near them. If they cannot find one then put 
one in the neighborhood of the last piece. Their scoring function for choosing 
the best place is $score_{dough,shapes}(m) = w_{11}n_{11} + w_8n_8 + w_5n_5$ and
they choose the best by finding the max of $score_{dough,myShapes}(m) - 
score_{dough,opponentShapes}(m)$. For their results they won most games against
groups 1,4,9. Also they created a really good destructive player as it created
much dead space and so the score was forced to be low.

Group 3:
Group 3 wanted to design a scoring function that will evaluate a given board.
But a generic search method to play a sequence of moves is too costly. So they 
choose to play the opening moves with a more loosely strategy and when the beard
has less moves then go their core strategy. For their function they wanted to be 
claiming space from their opponent and less for their pieces. Also space that 
cannot be claimed from the opponent should not be considered and favor larger 
pieces. Another main issue is to take into account interference between possible
moves. 
$\Delta f(C) = \sum\limits_{(i,j)\notin C} (X_{i,j}' - X_{i,j}) + \sum\limits_{(i,j)} (Y_{i,j}' - Y_{i,j})$ 
where $X_{i,j} \in \{0,5,8,11\}$ the size of the cutter. They had the best results
as they won 76 out of 80. They only lost sometimes to group 7.

Group 1:
The group did some experiments on its own and created some strategies that was
specific to counter act the strategies of specific players.


