% Use style package recommended by conference.
\documentclass{article}
\oddsidemargin=1cm
\textwidth=15cm

\usepackage[all=normal, paragraphs, indent, lists, floats, sections, title]{savetrees}

\usepackage{epsfig}
\usepackage{graphicx}
%\usepackage{subfigure}
\usepackage{subcaption}
\usepackage{verbatim}
\usepackage{amsmath,amssymb}
\usepackage{color}
\usepackage[hyphens]{url}
\usepackage{listings}
\usepackage[normalem]{ulem}
\usepackage{floatrow}
\usepackage{xspace}
\usepackage{amsmath}
\usepackage{enumitem}
\usepackage{booktabs}
\usepackage[utf8]{inputenc}
\usepackage{authblk}
\usepackage{hyperref}
\usepackage{algorithm, algpseudocode}
\usepackage{fancyvrb}
\usepackage{cprotect}
\usepackage{mdframed}
\usepackage{array}
\usepackage{cite}
\usepackage{multirow}
\usepackage{caption}
\usepackage{listings}
\usepackage[usenames,dvipsnames]{xcolor}
\newcommand\independent{\protect\mathpalette{\protect\independenT}{\perp}}
\def\independenT#1#2{\mathrel{\rlap{$#1#2$}\mkern2mu{#1#2}}}
\begin{document}
\title{Programming and Problem Solving:\\ A Transcript of the Fall 2015 Class}

  \author{ {\rm Kenneth A. Ross, Orestis Polychroniou and Georgios Koloventzos} \\ }
\maketitle


This report contains edited transcripts of the discussion held in Columbia’s 
Programming and Problem- Solving course, taught as W4444 during Fall 2015. 
The class notes were taken by the teaching assistant Georgios Koloventzos 
so that students could focus on the class material. As a service to both 
the students and to others who would like to get some insight into the class 
experience, we have drawn all of the class handouts, discussion, and some of 
the results into this technical report.
\clearpage
\tableofcontents
\clearpage
\section{Introduction}

\subsection{The First Handout: Course Description and Problem Statements}
\begin{center}
Fall, 2015\\
Mondays and Wednesdays, 1:10–2:25, Prof. K. Ross\\
688 Mudd Building
\end{center}

\subsection*{The Course}

This course is like no other course you've taken.  There are no exams.
There are no homeworks.  There are 4 projects, but all of the projects
are open-ended: there are no ``correct'' project answers.  The idea is
to develop problem solving skills using techniques that you have learned
during your CS training to solve the
project problems.  Lectures will be relatively informal, with most
of the class interaction being cooperative problem solving and discussion
rather than a fixed set of lectures with questions.

Some of the supplied problems may be too tough to make much progress on
in the 3 weeks you have.  If so, you should try to find special cases
of the problem that can be solved, or to simplify the problem in
such a way that it still remains interesting, but may be solved.

Conversely, some of the problems may be too easy.  In that case you
should try generalizing the problem.  To what other kinds of problem
can your solution also be applied?

For those of you familiar with \htmladdnormallink{Bloom's
  taxonomy}{http://en.wikipedia.org/wiki/Bloom's_taxonomy}, this class
emphasizes the upper levels (application, analysis, synthesis,
evaluation), and assumes the lower levels (knowledge, comprehension)
from your previous CS background.

All projects will be done in groups of three or four.  There are four
separate projects, and groups will change from project to project.
{\em You cannot be in the same group as another person more than
twice.}  We will hopefully be able to mix up the groups sufficiently
so that no pair of students is in the same group even twice.

A large proportion of your grade will come from class participation.
{\em Class attendance is mandatory.}  Do not sign up for this class if
you think you will need to miss classes.  An attendance record will be
kept by the TA.

This is the kind of course in which you can invest a little or a lot,
and for which your enjoyment and fulfilment will come in proportion
to the amount of energy you put into the course.

\subsection*{Prerequisites}
W3137 or W3134 (Data Structures),
and W3827 
(Fundamentals of Computer Systems).  
%W3824 (Computer Organization) can substitute for W3827.
Recommended: W4701 (Artificial Intelligence).
{\bf Permission of the instructor is required for all students.}
Senior
CS or Computer Engineering majors and CS graduate students only.
(Advanced juniors whose CS preparation is comparable to that
of seniors will also be considered.)
Enrollment will be limited to 30 students.

Permission to register for the class will be given by the instructor
{\em after the first class meeting}
according to the policies in the section ``Registering for the Course''
below.

\subsection*{Office Hours}
\begin{description}
\item[Instructor:] Prof.\ Ken Ross, 510 Computer Science Building,
212-939-7058, {\tt kar@cs.columbia.edu}

Office hours: Mondays 2:30--3:30pm, Thursdays 2--3pm.

\item[Teaching Assistant:] Orestis Polychroniou, 725 CEPSR, {\tt orestis@cs.columbia.edu}

Office hours: Tuesdays 11am--1pm.

\item[Teaching Assistant:] Georgios Koloventzos, {\tt gkoloven@cs.columbia.edu}

Office hours: Fridays 11am--1pm in the TA Room (1st floor Mudd).

\end{description}

\subsection*{Grading}

Projects will be graded by the instructor.  However, there is no
predefined grading scheme.  A grade will be given based on the
approach used to solving the given problem, including
\begin{itemize}
\item the novelty of the approach
\item the thoroughness of the project report
\item the clarity of the report presentation
\item the quality of the code
\item the correctness and generality of the proposed solution
\item the efficiency of the proposed solution
\end{itemize}

Several of the projects may have competitions to see whose program
``wins'' against other programs.  The outcome of these competitions is
not a factor in the project grades.  The competitions are intended to
stimulate the development of new ideas, and it is the ideas that are
important.  The competition rankings are for fun and pride only.

There is no restriction on communication between different groups.
However, if a certain idea X is proposed by group A, and group B
thinks the idea is so cool that they want to use it too, then group
B can do so {\em as long as group B explicitly cites group A's
contribution in their report.}  So, for example, suppose that
during the second class discussing project 1, a student Jane Smith reports
to the class
that she has discovered that a certain algorithm in the literature
applies to the problem.  Then other groups are still free to use that
algorithm too, but in their report they must include an ``acknowledgements''
section with text something like
\begin{quotation}
We acknowledge the contribution of Jane Smith who observed that the
17-starving-philosophers-Named-Josephus algorithm applies to this
problem.
\end{quotation}
Any information supplied by the instructor or
teaching assistant need not be
acknowledged in this way.  Published sources and
Web sources should be formally cited using a bibliography.

If a group develops a tool that is potentially useful to other groups,
then you are encouraged to share it (and it's use should be
acknowledged).  However, this sharing should be limited to tools that
aid the development of solutions (eg., visualizers, compilers, etc.)
and not the solutions themselves.  Similarly, external data that might
be useful should be shared.  For example, if we had a map-drawing
project, then if somebody downloaded from the Web some datasets describing
real-world maps that could be used to test project code, that data
should be shared and used with acknowledgement.  (For the purposes
of this course, the use of other people's contributions without
acknowledgement is considered cheating.)

Outside people may be consulted for general references, but shouldn't
be asked to contribute to solving the problems themselves.  This is a
course whose primary purpose is to give you the opportunity to
tackle challenging problems yourselves.
If you consult outside people, they should
also be acknowledged.

While groups are permitted to communicate with each other, they are
not encouraged to work so closely together that they really could be
described as an 8-person group.  Groups must submit their own reports,
and the reports should be written independently of other groups.

A report must also include a section titled ``Summary of
Contributions'' that describes (in one or two sentences per person)
what each member of the team contributed to the project.

Reports should also contain (where applicable) a paragraph documenting
the use of code written by you that is used in a substantive way in
the submissions from other groups.  After the final code deliverable,
you have the opportunity to browse through the other groups' submitted
code to identify such use.  Remember that this kind of use is
encouraged, and hopefully in most cases the other groups will have
told you directly that they are working with your code.  The purpose
of including this paragraph is to make sure that the instructor can
see where groups have had influence on other groups.

Reports should be submitted electronically, as well as in hardcopy on the
day of the group presentation.  The electronic reports will be made
readable by all members of the class after the submission deadline.

Here are the grading proportions:

\begin{tabular}{ll}
Project Reports and Presentations & 60\% \\
Class Participation & 40\%
\end{tabular}

All members of a group will, by default, get the same grade for their
project report.  In rare circumstances, the professor will adjust the
grade of an individual member of a project team if there is clear
evidence that this member did not make a genuine effort to work on
the project.

In addition, you are required to attend all classes.  You can miss
two classes without penalty.  However, the third class missed will
result in a drop in the {\em final grade} of one fractional grade
(eg., B+ to B, or A to A-).  If you miss four classes, you will
drop a full grade (eg., A- to B-).  If you miss five or more
classes, you will fail the course.  Arriving in class more than
15 minutes late counts as missing the class.  (In the event of
a medical or other emergency, consult the instructor.)


\subsection*{Policies}

There will be no extensions for projects.  Hand in what you have on the due
date.

An electronic newsgroup will be set up for all course-related
announcements.  Check there regularly for
any course related announcements.  Questions or comments of a general
nature that would be of interest to the whole class should be posted
there.  The instructor and the TA will be reading the newsgroup
regularly.

The web site for the course is at
\htmladdnormallink{http://www.cs.columbia.edu/\~{}kar/4444f15}{http://www.cs.columbia.edu/~kar/4444f15}.
The TA will be taking notes of our discussions during class, and those
notes will be posted on the TA page.  Links to
other relevant resources will be
put on the web page or on courseworks.

You can use any computing platforms you want as long as your programs
conform to the interface specifications described within each project.
You will probably need \htmladdnormallink{CS
accounts}{https://www.cs.columbia.edu/~crf/accounts/cs.html} for this
class.

The number of times that a single student may contribute during one class
is limited to three, in order to allow for as many participants as possible.

\subsection*{Tentative Timetable}

\begin{description}
\item[Sep 9] Class introduction and overview.
Brief discussion of all 4 project problems.  Students submit questionnaires.
\item[Sep 11] Announcement of selected students on the class web page.
HW for Sep 14: Read thoroughly the description of project 1.
\item[Sep 14, 16, 21, 23, 28] 
Formation of teams for project 1.
Project 1 discussion.
\item[Sep 30] Project 1 final code submission deadline.
Formation of teams for project 2.
Project 2 discussion.
\item[Oct 5] Project 1 reports and presentations.
\item[Oct 7, 12, 14, 19] Project 2 discussion.
\item[Oct 21] Project 2 final code submission deadline.
Formation of teams for project 3.
Project 3 discussion.
\item[Oct 26] Project 2 reports and presentations.
\item[Oct 28, Nov 4, 9, 11] Project 3 discussion.
\item[Nov 13] Project 3 final code submission deadline. 
\item[Nov 16] Project 3 reports and presentations. Formation of teams for project 4.
\item[Nov 18, 23, 25, 30, Dec 2, 7] Project 4 discussion.
\item[Dec 9] Project 4 final code submission deadline. Course wrapup.
\item[Dec 14] Project 4 reports and presentations. Peer-review submission. 
\end{description}

\subsubsection{Project 1: Parallel Pied Piper Plaza}
Last year we had a project called ``Parallel Pied Pipers'' whose
description can be found
\htmladdnormallink{here}{http://www.cs.columbia.edu/~kar/4444f14/node16.html}.
(You should read last year's description before continuing.)
This year we're going to use the same basic idea, but now it's a
multi-player game!  The shape of the field that initially contains the
rats is a square of size $F$ meters.  In the middle of each of the
four sides of the field is a 2m wide gate.  Each of the four teams of
pipers tries to get rats though their own gate (they are getting paid
per captured rat).  Once a rat goes through the gate it exits the
game.  There is a small holding area outside each gate, of size $F$
meters by 10 meters, where a piper can move if he wishes.  All pipers start
spaced out within the 2m gate.

When pipers are away from other pipers, or are near other pipers from
their own team, the piper/rat behavior is the same as last year's
project.  However, because different piper teams play different music
genres, rats are confused by the dissonance of different tunes.  In
particular, consider a rat that can hear the music of multiple pipers
from different teams because they are all within 10m of the rat and
playing music.  If there is a single dominant tune (e.g., one of the
teams has more nearby pipers than any single other team) then the rat
will be attracted to the closest of the pipers from that team.
Otherwise, the rat will behave as if there is no sound, e.g., the rat
will not be attracted to any of the pipers.

Each of the four teams has $d$ pipers.  Each piper is executing an
instance of the same code, but instantiated with a unique piper-id
parameter (between 1 and $d$).  Pipers have complete access to all
positional information, including the positions and music-playing status of
the other teams' pipers, but pipers do not explicitly communicate with other
pipers.

The simulator will detect when all rats have been eliminated, and you will be ranked based
on the number of rats that exited though your gate.  There will be a timeout to
handle cases where some rats simply can't be convinced
to go though any gate.  We will run some tournaments at the end of class
with various values for the parameters $F$, $S$ (the number of rats), and $d$.

You will have access to the players and reports from last year's
teams.  You can choose to use code written last year if you wish,
making sure to make suitable acknowledgements.  Alternatively, you can
write your own code from scratch if you prefer.

Some ideas to think about:
\begin{itemize}
\item What kinds of coordinated teamwork (if any) are needed in this multiplayer version of the game?
\item What is the impact of the dissonance of music from multiple nearby pipers?
\end{itemize}

\subsubsection{Project 2: Planet Builder}

Asteroids of equal size orbit the sun in initially circular orbits at
various radii.  There is a lower and upper radius bound, so that
the asteroids are in a formation like the asteroid belt between
Mars and Jupiter.  You get to
push the asteroids a bit to change their trajectory.  When asteroids
collide there is conservation of momentum and the asteroids remain
joined (and larger).  Energy is not conserved: the newly formed
asteroid heats up!  Your goal is to create a single planet made up of at
least half the initial asteroid mass within a fixed time $T$.  Each
time you exert a push, the energy required for that push is
accumulated into your score; you want to minimize the total energy
used. (Using a lot of energy to form a planet quickly scores worse
than forming the planet more slowly with less energy, as long as the
planet forms within $T$ time units.)

Asteroids are influenced in their trajectory by the sun, whose gravity
is strong.  The gravitational attraction of other asteroids is assumed
to be negligible and is ignored by the simulator.  All interactions
(including your pushing of asteroids) happen in the plane of rotation
of the solar system.  A push perpendicular to the movement of the
asteroid will change its orbit, but sill retain the circularity of the
orbit; a non-perpendicular push (or collision) may lead to an
elliptical orbit.

For a cool related game concept, see \htmladdnormallink{Osmos}{http://www.osmos-game.com/}.
\subsubsection{Project 3: Cookie Cutter II}

In 2003, I gave a \htmladdnormallink{project}{http://www.cs.columbia.edu/~kar/4444f03/node16.html}
involving cutting cookies from a piece of dough.  This year we're doing something different: competitive
cookie cutting!  This is a game for two players, played on a 50x50 grid of dough.  Each player designs three
cookie cutter shapes (details below) and tries to cut out as much of the dough as possible before the
dough runs out.  The score is the total area of dough that your player manages to capture.  Your primary
goal is to capture more dough than your opponent (as opposed to maximizing the amount of dough), so
defensive strategies that block your opponent while blocking yourself to a lesser degree are worth
considering.

You get to make three cutters of size 11, 8, and 5 units.  These
cutters must be a connected orthogonal set of unit squares.  So the 5-unit
shape must be one of the 12 possible
\htmladdnormallink{pentominos}{http://en.wikipedia.org/wiki/Pentomino},
for example. (Actually, since we don't consider reflective symmetries, there
are more than 12 choices: how many are there?)  Each player creates their 11-unit shape first and
submits it to the simulator.  The simulator shares this information
with both players, who then select their 8-unit shape.  After the
simulator informs both players of the 8-unit selection, players submit
their 5-unit shapes.  So players get to design the smaller shapes in
light of the known choices of both players for the larger shapes.

One final rule about selecting shapes.  The simulator will not allow players to select
the same shape (taking the four rotational symmetries into account).  If players
happen to select the same shape, both players' choices are rejected and players have to
select a new shape, different from all previous choices.  If the simulator has to reject
players' shapes 5 times in a row, then the simulator will randomly assign one of the five
choices to each player (a different shape to each).

The game is played by placing a cookie-cutter of your choice anywhere
on the board in one of the four orthogonal orientations.  You may only
select a location if there is dough remaining within each cell
delimited by the cookie cutter.  Note that the cookie-cutter cannot include partial dough
cells: the perimeter defined by the cookie-cutter must lie on the ``grid lines'' of the dough grid.
At the end of the turn, your score is
incremented by the size of the cutter used, and the corresponding
dough is removed from the grid.  Players take turns, but to balance the
game the first player must use the 5-unit cutter on the first turn.

Some things to think about:
\begin{itemize}
\item What are the benefits and risks of regular shapes compared with irregular shapes?
\item Near the beginning of the game, how should you space out your choices?
\item How does knowing your opponent's larger shapes help you select your smaller shapes?
\end{itemize}
\subsubsection{Project 4: Work the Room}

You're planning to attend a big party.  Some of your friends will be
there, but many people will be strangers.  Among the strangers is your
unknown soulmate.  Your goal is to have beneficial conversations with
as many people as you can for three simulated hours. Friends are
interesting to talk to, strangers less so, but your soulmate is
particularly interesting.  Also, you need to move around in order to
find people and chat with them.  The party takes place in a large
square room measuring 20 meters on each side.

Each turn of the simulator corresponds to a 6-second chunk of
real-world time. During any turn your player may attempt to do one of the following:
\begin{itemize}
\item Move to a new position within the room anywhere within 6 meters of your current position. Any conversation
you were previously engaged in is terminated: you have to be stationary to chat.
\item Initiate a conversation with somebody who is standing at a distance between 0.5
meters and 2 meters of you. In this case you do not move.
You can only initiate a conversation with somebody on the next turn
if they and you are both not currently engaged in a conversation with
somebody else. (So there has to be at least one turn between conversations.)
\item Continue a conversation that you were participating in during the previous turn. In this case you also do not move.
\end{itemize}

All players submit their orders together, and the outcome is resolved by the simulator using the following rules in sequence:
\begin{enumerate}
\item If two players were previously chatting, and both wish to continue the conversation, then
the conversation continues.
\item If two players, who were not previously chatting to anybody, simultaneously try to initiate a conversation
with each other, then a conversation between them is begun.
\item For each remaining player who tries to initiate a conversation, figure out how close each is to
their conversation target, and do the following in increasing distance order (with ties broken randomly):
\begin{enumerate}
\item If the target is not already engaged in conversation on the current turn, then begin a new conversation
between the two players.  The target cannot refuse (you have to be polite!) but can terminate the conversation on the
next turn.
The target's action for the current turn (whatever the action was) is cancelled and the target stays put.
\item If the target is already engaged in conversation on the current turn, then the player stays put and does
not begin a conversation.
\end{enumerate}
\item Players who try to continue a conversation with a partner who chooses to move away
just stay put, and the conversation is ended.
\item Any remaining players must be attempting to move.  All of those moves succeed.
This includes players who terminate a conversation by moving:
no other player would have been allowed to start a new conversation with them.
\end{enumerate}
Illegal moves (e.g., moving more than 6 meters in a turn, trying to start a conversation with somebody too close or
too far away, etc.) will be treated by the simulator as a ``stay put'' directive.

The purpose of a conversation is to accumulate
gossip/wisdom/life-lessons, that come conveniently packaged in
6-second units.  The wisdom that players can offer each other is
captured in a two-dimensional array $W$.  There are $N$ players in the
game, where $N$ is a parameter, so $W$ is an $N$ by $N$ array where
$W[i,j]$ represents the total wisdom that player $i$ has to offer
player $j$. ($W[i,i]$ is not meaningful.)  It is not always the case
that $W[i,j]=W[j,i]$. 

On a turn when players $i$ and $j$ are chatting, each has the
opportunity to collect a unit of wisdom from their partner.  However,
they will only obtain this wisdom if there is no other ``interfering''
player closer to them than their conversation partner (too hard to
hear otherwise!). It therefore pays to try to start your conversations
away from other players.  Note that it is possible for one
partner in the conversation to gain wisdom, while the other does not
due to interference.  If $W[i,j]$ is nonzero and nobody is closer to
$j$ than $i$ then the score of player $j$ is incremented by one and
$W[i,j]$ is decremented by one. A symmetric outcome happens for player
$i$. Once $W[i,j]$ reaches zero, $j$ has no further incentive to chat
with $i$.

Players have special relationships with other players.  Each player
has $f$ friends, $s$ strangers, and one soulmate among the other
players, where $f+1+s = N$. Players know who their friends are, but do
not know who their soulmate is.  Friendship is symmetric.  If $i$ is a
friend of $j$ then $j$ is a friend of $i$ and $W[i,j]=W[j,i]=50$. In
other words, friends can chat productively for up to 5 minutes.

Soulmates are also symmetric, but with $W[i,j]=W[j,i]=200$.  If $i$
and $j$ are neither friends nor soulmates, then $W[i,j]$ is chosen at
random from $\{ 0, 10, 20 \}$. To balance the game, the simulator will
make sure that each player has the same number of $0$, $10$ and $20$
values among all strangers. For nonfriends, when player $i$ chats with
player $j$, player $j$ learns $W[i,j]$ and player $i$ learns
$W[j,i]$. Prior to that initial conversation, $W[i,j]$ is unknown.
Player $i$ is unaware of the values of $W[i,j]$ for strangers $j$.

Finally, your player can see only a radius of 6m around its current position.
The simulator will tell you the positions of all players within that radius.
(Your player is free to remember previous locations of players that have
gone outside the 6m radius if that
information would be helpful later.) If two players who are both within
6m are having a conversation, the simulator will make your player aware
of this conversation. (How could that information be helpful?)
The values of $N$, $f$, and $s$ are known to each player at the start of the game.

We'll run several tournaments at the end of class using various
parameter settings for $N$, $f$, and $s$. Large values of $N$ are
possible, in which case we'll use multiple instances of players from
each group.

\subsection{Participants}

The Names of participating students are listed below.  If you are listed but do not intend to
take the class, please let the instructor know ASAP by email, so that
he can determine whether to accept students from the standby-list.
(Members of the standby-list have been notified by the instructor via email.)

\begin{tabular}{ll}
Name & Email (@columbia.edu unless specified) \\ \hline
Sania Arif & sa3311 \\
Alice Chang & avc2120 \\
Manyi Chen & mc3962 \\
Amar Dhingra & asd2157 \\
SAmarth Dhingra & sd2900 \\
Preetam Dutta & Preetam@cs \\
Vinay Gaba & vinay.gaba \\
Rhea Goel & rg2936 \\
Jing Guo & jg3527 \\
Xingzhou He & x.he \\
Nam Hoang & nnh2110 \\
Konstantin Itskov & koi2104 \\
Naman Jain & nj2303 \\
Cathy Jin & ckj2111 \\
Tingting Li & tl2617 \\
Sean Liu & sl3497 \\
Parthiban Loganathan & pl2487 \\
Diana Liskovich & dl2956 \\
Seth Mishan & sm3890 \\
Ananya Poddar & ap3317 \\
Dhruv Purushottam & dp2631 \\
Artur Renault & Artur.renault \\
Sagar Sarda & ss4355 \\
Kevin Shi & kshi@cs \\
Vishal Vyas & vnv2102 \\
Liuyang Wang & lw2635 \\
Robert Ying & ry2242 \\
\end{tabular}

\subsection{Project Teams}

The following assignments were made during the course of the class. Groupings were random, 
except that people who had previously worked together were (as far as possible) not 
grouped together for subsequent projects.
\\
\\
Project 1
\begin{enumerate}
\item Kostantin - Parthiban - Sagar
\item Kevin - Preetam - Naman
\item Amar - Dhruv - Sean
\item Tingting - Ananya - Vishal - Rhea
\item Sania - Vinay - Seth
\item xingzhou - Robert - Alice - Nam
\item Manyi - Diana - SAmarth
\item Jing - Artur - Cathy - Liuyang
\end{enumerate}
Project 2
\begin{enumerate}
\item Preetam - Vinay - SAmarth
\item Sean - Derek - Parthiban
\item Robert - Cathy - Manyi
\item Dhruv - Artur - Alice
\item Amar - Sagar - Vishal
\item Konstantin - Tingting - Nam
\item Kevin - Ananya - Diana
\item Naman - Sania - Jing
\item Rhea - Seth - Liuyang
\end{enumerate}
Project 3
\begin{enumerate}
\item Konstantin - Preetam - Vishal
\item Sagar - Dhruv - Vinay
\item Derek - Kevin - Sania
\item Artur - Sean - Nam
\item Ananya - Amar - Jing
\item Parthiban - Robert - Diana
\item Naman - SAmarth - Liuyang
\item Cathy - Alice - Rhea
\item Manyi - Tingting - Seth
\end{enumerate}
Project 4
\begin{enumerate}
\item Alice - Sean - Vishal
\item Dhruv - Parthiban - Kevin
\item Artur - Amar - Preetam
\item Cathy - Nam - Diana
\item Tingting - Jing - Sagar
\item Manyi - Naman - Ananya
\item Liuyang - Konstantin - Vinay
\item Rhea - Robert - Sania
\item SAmarth - Derek - Seth
\end{enumerate}

\section{Parallel Pied Piper Plaza}
\subsection{Monday September 14, 2015}
First students asked questions about the project, the summary is:

Groups will have to write code for a player. Number of rats, pipers and the (square) 
board dimension will be dynamic as arguments (Read the README file and the sample 
code that are in the simulator).

General:
You will now all the position of pipers/rats.
Gates are always in the middle of the edges for each player.
The pipers will be distributed evenly inside their box as starting points.
They all have to get out of the gate.
If you specialize your solution that is fine. You have to explain it in your presentation.
Throw exceptions! Do not exit!

Rats:
They will follow the dominant piper. If more than one of the same player will 
follow the closest.
For the direction of rats you should calculate it from the 2 consecutive runs. 
You can save states for such calculations. 
If rats enter through the gate cannot come back (as opposed to last year’s project).
If they get confused by dissonance of different tunes, they get a random new direction.

Pipers:
Each piper has upper bound speeds for playing and when not playing. Pipers / rats can overlap between them. 
Pipers cannot communicate between them but they can know what strategy the other piper is using. 
You will know if a piper was playing music (or not) in the previous run.

Tournament:
The player that gathered most of the rats is winning the tournament (but not a 
better grade).
Many tournaments between random teams will be held. The best team will be the 
one that gets the same amount of rats (the bigger amount the better) at 
tournaments. 


Strategies:

Dhruv stated that having a piper as magnet as done for the last year’s project 
will not be feasible as now a rat can be stolen from another team.
Preetam asked if fighting between the players is viable. Professor said that you 
can do it but it may not be the best practice as you can compete with more pipers.
Kostantin suggested that we should abandon rats that are far away from out gate. 
That rats will be easily attracted from pipers whose gates are closer to them. 
Jake started the discussion about disruption of other layers. In particular will 
be nice to have a piper to disruptive other players. This can be in vain if more 
than one piper per rat.

A method that can be transferred from last year’s project is the optimizing the 
movements of our pipers with trajectories of rats. 

Tingting proposed to divide the space and direct our pipers to specific location.
Ethan continued that in last year’s project were solutions about dense versus sparse 
spaces of rats. Professor added the solution of swipe for dense spaces.
Parthiban asked if there will be any time or space limit. Professor explained that 
there will be a time limit. If this limit is reached, it means that there still 
are free rats. The tournament ends and we do not take into account the rats outside holding areas.
Derek wanting to know what would be the numbers of the tournaments. Professor answered that a number of 10000 rats and 10 pipers are feasible. 
Vinay added that the coordination between pipers can be transferred from last year’s project but with some changes.
Sagar started talking about disruptive tactics with the idea of positioning a piper 
in rivals gate. Sania replied that having 2 pipers controlling the rat will be 
easy to pass such problem. Manyi suggested that you can leave yourself one in your 
gate and that will solve the problem. Ethan continued the idea of doing the same 
in the rival’s gate. Konstantin pointed out that if you have small amount of 
recourses you can harass one or two player maximum. Some other student pointed 
that if you have more than 2 rivals is not a good strategy as you are losing 3 
pipers that can get you some more rats (if you know you she/he was email me).
Naman suggested that we can sabotage the best player. Kevin added some corner case maybe are worth sabotaging like 10 rats following one piper. 
Manyi recommended that we can send all pipers to one rat. Sania added that this strategy will have low coverage.
Jing Hu suggested that we should find the region with more rats and head there (density).
Alice proposed that the player should attract first the rats nearby your gate. Naman said that it will better to get the ones further as the closest ones will be easy. Venkata urged that the closest ones will change by the time you will have come from the ones that are further.
Rhea was the one first suggested that all of the strategies have some cost and would should take into consideration such fact. IF you send 2 pipers the cost of recourses are double.
Jing Guy’s strategy was to send “bodyguards” if a piper has many rats following him.
Jake points out that we can benefit if we can trace rats that are coming to our direction.
Professor explains the last year’s solution with the team that was making the piper play until the rat had the direction that will lead him to an exit. Also told why this is not a good strategy for this year as the rat can be stolen. 
Manyi the risk on the bodyguard strategy as long the bodyguard(s) have not arrived yet and the rats can be stolen or lost. 
Samarth asked if the pipers can know what the other pipers are doing. Professor said that as this can be hardcoded at the beginning is feasible (all mod 3 pipers are having the same strategy). 
Vinay’s strategy was that we should divide and distribute the pipers in the plane.
Ethan suggested that a team can track the pipers position and understand if they are sabotaging/swiping/marauding in order to adjust theirs strategy.
Tingting added that a piper can take the long way to the gate than the shortest. With such strategy can avoid pipers and probably swipe some space. Professor named it Little Red Riding Hood strategy.
Parthiban added last that we can harass the player with the most accumulated rats.

The deliverable of this course will be a sample code for player that has a feature.
E.g. 2 pipers in coordination.

\subsection{Wednesday September 16, 2015}
Today we run from all teams their deliverables and we discussed about their tactics.
Most of the runs was with 4 pipers -100 rats. Unless it is specified.

Group 1: 
The group divide the plane in smaller cells and directed their pipers into the more dense one.
At the end all pipers were fighting over a single rat.
Professor stated that this will reach the timeout. 
Preetam noticed that some pipers randomly left.
Amar stated that the cost of going far away from your gate is too big.
The group responded to Amar that this strategy will also swipe the are from the dense cell to the gate as they return.
Sania told that going far away creates risk of losing rats as you return.
Dhruv asked about if it will better if a proportion of rat and pipers per cell will be better.
Group responded that can lead to stolen rats as may some other team will have more pipers


Group 2 (3 pipers):
The group has a strategy of assuming that rats have positive charge and the pipers negative (or the other way around). That means that pipers are not closing together but they are attracted to rats. Also they are trying to lure the closest rats first. 

A student comments that the local optimization may not be the best as a small number of rats will be close to gate (I missed the name).
Konstantin asks if the group has changed the after the gate movement of the pipers as they were losing some that came to the gate.
Group responds as they do not change it.
Ananya stated that the repulsion between pipers will be a disadvantage as the bodyguard strategy.
Preetam told that the return to the gate is hard coded in order not to have repulsion between the pipers.
Robert also suggested that if there are a small amount of rats it would be better to drop the repulsion as it will be a fight between a number of pipers from each team.

Group 3 (5 pipers):
The group have implemented a sweeping strategy. The pipers are positioned in specific locations and they are coming closer together as they getting closer to the gate.
Preetam spots that even when no rats are being lured they still return to gate.
The response of the team, Amar, was that is a prototype.
Sean stated that this will be good strategy for dense cells as the are minimizing losing rats close to gates.
Sagar asked how they decide where the pipers will be positioned.
Amar replied that they are position at $\frac{2}{3}$ of the distance of the whole arena.

After this example a discussion started about how they should categorize a space as dense and when not.
Alice pointed out that they should run many times their experiments and decide when the space of the game is dense and when it is sparse. 
Professor added that would be something to analyze in your report with graphs/pictures. This good give you an insight when and if you want to change tactics between dense versus sparse arenas.
Liuyang states that we should add and how many pipers exists, to the density measurement 
Cathy adds that if you have less pipers and more rats in a cell you should start team up your pipers


Group 4 (12 pipers):
The group divides the arena into 4 cells and scans it vertically first and then horizontally.
Konstantin states that is a good strategy for many pipers.
Tingting adds that their strategy is not aware of rats positions. They followed this algorithm in order to get many rats at the first iterations.
Sanya asks how the sweeping is done.
Ananya answers first vertically then horizontally.
Parthiban adds that is a good strategy but they should calculate if there are any rats in the sweep lane otherwise a piper is wasted.

Group 5 (1 piper):

The team asked with one piper as they did not have tested it with more pipers.
Their strategy is to find incrementally the closest rats near their area and quick lure them inside their box.
Dhruv states by doing that they can lose rats from marauding pipers.
Arthur states that if they are no rats in the closest they will be exposed into stealing as one piper goes for each rat.
Kevin spots that some rats are left. This happens because they were not taking into account the reflection in walls as rats are moving.
Preetam states that is a good algorithm for sparse areas but it will lose in a dense one.
Manyi adds that aid should be given if someone ends with lots of rats.

Group 6:
The group is forming teams of pipers and sweeps areas

Artur spots that there is a trouble at the gate as pipers are stucked for some seconds. 
Group responds that this is due to small gate and many pipers.
Robert suggests that we can found how the groups are formed and counteract.
Professor states that this can lead to escalation as all of the teams will try to overcome some opponent and finally all pipers will be as one.
Alice raises the issue that as you are waiting to make a decision you will lose valuable time and that can cost you rats especially at the first turns.
Jake specifies that staying in small groups and avoiding other groups can make such strategy better and avoid marauding pipers.
Preetam states that we should not steal for the sake of stealing. Trying to steal will cost you some pipers that can be bodyguard some other piper to give you better outcome.

Group 7 (1 piper):
The team sweeps the densest area.

Professor asks what are the problem with this strategy.
Diana answers that the tail of the piper is vulnerable 
to stealing. Because as the piper will detect that a 
dense area is near the other player will go to play 
near the tail of rats the other player creates. So the 
furthest rats will get confused, change into random 
directions and at the end they will become unattracted.
Professor states that if you are in such situation you can either go with 
less velocity or have stops in order not to have such trail of rats.

Group 8 (3 pipers):
The group sweeps the denser area to get as many rats as it can get. After there they are clustering the pipers to get some sparse rats.

Group 9 (2 pipers - 10 rats):
In their strategy every piper tries to steal back any rat that got lost.

Group 10 (5 pipers):
They are also coded the pipers to start sweeping and start closing at the gate.

Important notes from today:
Naman: A dense strategy is good for the beginning of such maps.
Derek: Getting more rats at first runs is crucial.
Alice: Coordinate your strategies between dense and sparse maps
Amar: Creating a factory of strategies to change during the game.
Parthiban: The game is presented as a minimax algorithm of artificial intelligence. We should find where we should position our pipers to get maximum rats. 
Cathy: First we should look for rat density and after some runs we should be caring more for piper density.
Sania: We have to understand and compute the cost between stealing and sweeping


Deliverable for next class:
Continue your code and next time we will have some tournaments between groups.
The tournaments will be done with dense map (4 pipers 100 rats) and in sparse ones (4 pipers 10 rats).

\subsection{Monday September 21, 2015}
The course today was tournaments with the deliverables of each team. We have agreed upon 2 configuration for today. A dense configuration with 4 pipers and 100 rats and a sparse with 4 pipers and 10 rats. 

First tournament with dense config amongst groups 9,8,7,6:
Group 9: Group 9 has a dense strategy that sends their pipers as a triangle to sweep the area and converge to their gate. When the area is sparse they are using a divide and conquer approach of sending pipers to closet rats.
Group 8: Group 8 is doing 3 sweep passes of the area in stripes and then are locating the denser areas and sweep them.
Manyi: We divide in 16 cells and we sweep the denser ones.
Samarth states that this strategy can help them steal rats from groups that are close to the gate and do not have many pipers.
Group 7: Their strategy is to sweep in stripes. After this they change to divide the pipers and some will go to the denser areas and the rest will try to collect rats that are close to their gate.
Group 6: Group 6 has a different sweeping strategy. Instead of converging to their gate the pipers are converging to center of the area. Then they are all heading to their gate. After this they will try to steal some rats from the rest.

Seth states that the sweeping strategy from group 7 was great but as they did not overlap they lost many rats before they converge.
Preetam adds that sweep techniques must take into consideration where are the other pipers and avoid them in order to lose less rats and probably cover/overlap to other areas.
Konstantin says that if we introduce a reactive strategy we must react to the movement of all pipers and not only to single ones.
Sania states that the change of strategy should be dynamically otherwise can lose time and rats by using a strategy that is wrong.
Professor adds when the area are sparse probably will be too late to achieve victory.So the groups should have different strategies if the map starts as sparse than becoming sparse after a dense tournament loses most of its rats.
Samarth claims that predict density can be dangerous for deciding where the pipers should go.
Rhea adds that if the group changes the granularity can avoid such problem.
Professor continues that in such a dense place many teams will end up to the denser area.
Ananya states, most of the movements should change dynamically. Going into a specific area can lead to a dense area that is no longer dense.
Dhruv adds that if the destination is far away it is more likely that the area may not be as dense as before.
Amar continues that no one is creating a return path through dense areas. This can lead to some more rats in the way back.

2nd tournament Groups 5 -4-3-2:
Group 5 has a strategy that takes into consideration the piper as a cell. So if a piper becomes dense pipers are going to assist him.
Group 4 explains a more elaborate circular sweep with a converge point near the center and then returns to the gate.
Alice states that their group has problems with testing as they always was against themselves.
Professor states that from now on they will have 8 players that can play against each other.
Sagar states that if we do not have many pipers this radio sweep could lead to nice results.
Cathy starts a conversation about where the pipers should be started playing in the area.
Rhea answers that they had run some experiments with the pipers in different places and they decided that the best will be in the middle. Further away will have more chances to lose rats. When they were closer they did not get enough rats. 
Vishal states a better strategy would be to dynamically converge when you reached a threshold of rats for guarding them.
Derek asks if the pipers are playing while they are going to their positions and the answer was no.
Robert said that avoiding the pipers of the other teams would be a better tactic than going straight to the gate.
Tingting says that a strategy they are thinking is to have a smaller group of returning pipers if no enemy piper is following.
Konstantin states that in general a piper for putting the rats inside is essential as they were having problems putting the rats inside their area.

Group 3 have created a strategy to counter attack sweeping techniques. They lost 
because most of their opponents converged earlier than they had thought.
Amar says that they were hunting the closest rats when the area is sparse. But 
they should estimate the cost of sending a piper for stealing rats.

Group 2 expressed that they robbed most of their rats after the sweeping strategy 
because they re-assigned one of the guard pipers. 
The problem of leading the rats through the gate was stated again. Orestis answered 
that pipers do not have any issue for being at the same place at the same time. 
So it should not that of a problem.
Vishal says that we should wait a bit before entering the gate in order to have 
the rats closer.
Naman adds that having the rats in straight line may be be better as you can enter 
the gate effortlessly.

3rd Tournament Group 1-4-8-5:
Sparse configuration.
Group 1 tries to steal with all of the pipers in each rat.
After this tournament many questions arose about stealing. Orestis said that you should not play music early as you miss the a part that you can lure the rat to you.
Sagar suggests that the dense and sparse should be cluster case and evenly distributed rather than a dense and sparse map. With such approach you can choose where your pipers should go. 

The deliverable of the next time should have a robust strategy. You should specify what robust means to you and apply it.
Students voted that the configuration should be chosen by the professor and the TAs.

Student should have in mind that the area may change.

Next important thing:
Vinay said that we have seen when only one rat is left, even we fight we will not change the outcome.
Sean claims that toggling the music on and off will give us better chance of stealing a rat in such situations
Nam added that if we all use the best technique for the last rat, it will be sheer luck who we will get if we do not hit the time limit.

\subsection{Wednesday September 23, 2015}
Today we will run tournaments with the new players of the groups.
The configuration was the convenience of the professor and the TAs.

Our first configuration was a bigger board of 300 with 2 pipers and 200 rats.
Starting with the groups 1,2,3,4 the gui was unresponsive. The problem was that group 2 was talking long to take a decision. The team specified that their algorithm has runtime quadratic complexity on the size of the board.

Removing group 2 and start with group 6 the gui crashed because of the too many informations that needed to be rendered. Because of this we change the size of the board on 200. Group 2 was slow even for this size.

After some tournaments Robert states that the strategies of the teams are similar. Not exactly the same but most of them are doing sweeping at the beginning and try to steal after some sweeps. 
Druv stated, most of strategies are trying to maximize how many rats are taking in the beginning.
Derek observed that at stealing we have many fights of different strategies.
Anaya says that only one team was toggling the music when stealing.
Artur answers that was an after effect of their strategy as when a piper loses his rat(s) stops playing music and tries to win it again.
Konstantin states that they stop the piper place him near the end of it circumference and start playing again. They are doing this if the rat is near the end of the circumference of the other piper they can steal them if the random direction, the rat will follow are towards the,/
Vinay says that splitting is more productive as the group coner more and can lure more rats to their gate.
Sania observes that the randomness of the rats are affecting the outcome as some 
times more rats are in the area close to the gate of a group.
Professor states this will be insignificant because of the many times we will run the tournaments
Tingting thinks that we have some space for improvement on the strategy we are 
following when we are returning to the gate.
Konstantin says that they did some experiments in order to find where is the optimal 
places the piper should go at the beginning. They found that the center is the 
best approximation.
Vishal says that the positions should be more dynamic as may most of the rats may 
be in the other side of the board.
Amar states that is good to split up early in the game to cover more area but the 
time of converge will maximize your rats. If you choose to converge later you may 
lose rats near the gate. 
Jing says that going for more rats at the beginning and try to control more rats 
before you converge is good strategy. And keeping the rats closer to the piper 
increases the possibility to actually return to gate with more rats.
Vinay says that his team has a semicircle that when a piper enters there more 
pipers are converging to bodyguard him.
We run some more tournaments with a different configuration. We increase the rats 
to 500 and the pipers to 6. Group 7 had an issue with the number of rats. Are more 
and more rats were out of the board the speed was increasing. 
Parthiban observed that many pipers were changing direction in the middle of a 
move. Such movements could cost time to the pipers and eventually rats. We have 
to do better prediction and try to minimize the lost time of such movement.
Sean says that when everybody is sweeping at the beginning of a dense board, it 
is safer to put all the rats with one piper and the rest should start sweep again.
Preetam continues that if you use such strategy, the piper should be close to the 
gate because if this is found by stealing groups, the piper should not be far away 
as they can stop playing music and go to the gate to steal from there.
Cathy says that we should discover if a team sweeps not in unison and try to 
steal with a sweeping team of more pipers.
Kevin states that their strategy dynamically founds how many pipers of a team 
tries to attack them and they send a +1 piper to bodyguard.

Here the conversation was pivot on what configuration the tournaments should run.
Professor asks if 1 piper is something worthy to test.
Vinay states that with one piper we should not focus on stealing.
Dhruv says that with one piper we will clean your area close to the gate fast and 
then it will take more time to find and get a rat.
Sania thinks that as most of the strategies are focusing in the teamwork maybe is not the best configuration. 
Amar thinks that with 1 piper we can distinguish which team has the best approach on how to prioritize the rats.
Kevin suggests that with 1 piper will be focused on your local triangle rather than going far.
Professor argued that the many pipers are producing more sophisticated strategies. 
A single piper can give us hints on how simple strategies can overcome others.
Sagar suggests that with one piper we cannot have any defense. So destructive 
players may be in favor.
Professor asked if we should introduce a maximum number of pipers.
Robert suggested that the maximum pipers would be the number when teams of 2 pipers 
can sweep all the board.
Vinay says that the number of rats should be the upper limit of the pipers. We 
should always have more rats than pipers.

The number of rats did not created much of a conversation as the student concluded 
that both sparse and dense boards should be covered.

The board size was suggested by the student not to be enormous.
Derek said that a big board will change the strategies as the pipers have too 
much distance to travel.
Sagar disagreed as all of the pipers will have the same speed. The problem will 
be when we will have a big board with few pipers. As the covering space of the 
playing pipers will not cover much of the board and the sweep will not be efficient.
Konstantin added that with big board the time of a fight for some rats will be 
bigger so it will be better to do other things instead.
Vinay suggested that if you are trying to do clustering then the big board will 
create a mees of such strategies.

The deliverable for next Monday will be an almost ready piper.

Cathy asked when will be the due date for the report and professor stated that 
it will be the same as the presentation. 

The presentation that will have more data and more clear explanation of why you 
choose the strategy will have better grade.
Every team will have 7 minutes to present so do some dry runs.


What is the most important thing for the next class:
Sagar thinks that we should optimize the return path as we can pick some more rats 
as we return.
Preetam suggested that we should avoid simple sweeps as it always lose rats.
Manyi thinks that we should focus in the efficiency of our strategy at the beginning 
and not in stealing
Sania thought is the point in time that we will choose to converge our pipers is 
crucial. We should investigate the trade off and pick the best time. 
Tingting opinion is that still the strategy lacks of sophistication and prediction.
Rhea raises again the issue that a better threshold when we change strategies are crucial.

\subsection{Monday September 28, 2015}
In the beginning professor stated how the presentations will be and what the report should look like.  For more have a look at the webpage.

First we had some question for the presentations and the report. 
Seth asked how long will the presentation will be?
Approximately 7 minutes per team.
Konstantin wandered if everybody should speak. Yes!
Sagar made a question if presentation should be in powerpoint.
You can use whatever seems fit from Powerpoint to google slides.
The TAs will help you with their laptops. 
Samarth was anxious if they found a bug after the deadline.
In general you can add this in your presentation but still we will take the bug 
into consideration
The results of the tournament will be published in a MySQL database. Some example 
queries will be given. Also a dump of the db will be also available.

Then we started with some experiments. First we started with a sparse board. 1 
piper with 10 rats at default size board.

Amar saw that 2 teams started fighting early and the other 2 lured the rest of 
the rats without big trouble.
Dhruv find out that many teams are trying to reclaim stolen rats that can lead 
to lose more rats in the process.
Preetam said that the rats that are dispersed when they lose the piper can be 
reclaimed or swept away easily.

Seth described how their team chase some rats that are close to the piper as they 
are returning to the gate.
Vishal points that this is a good strategy and they have implemented something similar.
Cathy points out that some teams try to sweep the board even though few rats were 
on the board. That led them far away from the gate and eventually with fewer rats.
Kevin suggests that going in a sparse board for rats far away is waste of time. 
In that time you can sweep your area and “secure” the rats close to you.
Lingyan observed that group 6 lost too many rats as they were returning.
Nam said that their team is good at stealing and we saw this in the experiments.
Robert adds that the toggling of music is crucial for stealing.
Derek states that if you are trying to steal and the rats goes away from you toggle music.
Sania described how such pipers are risking going far away.
Preetam confirms that going far away is not the best practice.
Konstantin explains how their distance function lead the pipers to go for a fight. 
A rat that was already lured was closer from a free one.
Sagar believes that even if they were going for the free maybe another piper will 
come to fight them. That means the result will be the same.
Manyi suggested that we should look on the distance of the rat and decide which 
one is easier to lure.
Rhea suggested that going far away can lead to tail problem.
Amar explains how the tail problem could be something good. Because the enemy 
pipers can take some rats from the tail and leave the piper with the big cluster.

Tingting claims that having a collection of strategies and then a meta strategy 
that can orchestrate them. But changing too many times or even once can lead to 
worse results.

Here we changed the configuration of our experiments. We went to a dense board to 
see how the players react. 1piper still with 500 rats and default size board.
Naman observes that still some teams are not efficient as they lose some rats 
because they are not going throught the gate.
Seth suggested that going for the rats closer to you may lead in some quick sweeps 
but it is not guaranteed that there will be rats.
Sania noticed that even with many rats some are choosing to fight instead of go get the free ones.
Vinay points out that you can afford to lose some rats to achieve that the rest will go through the gate.
Dhruv’s opininon is that the crucial points is when or not to sweep.
Ananya points that the sweep should be done in ares with more rats and less pipers.
Naman believes that sweeping fast and not far away form the gate and then try to 
steal find unlured rats and bring them back.

Being reasonable democratic the class decide that the configurtion for the tournaments should be:
pipers: 1- 6- 13 == 1 for testing this corner case, 13 to break symmetries and 6 to enforce them
rats: 20 -101 - 500: == sparse -medium dense
board: defualt and double the size

Last changes:
Artur believes that tuning the the thresholds will give you the upper hand
Dhruv suggests that now we know the configurations should we tune for them
Preetam said that we should minimize our pipers as targets. If there is an advesary piper
you should minimize how dangerous you are.
Vinay proposes that changing strategies can be rewarding.
Ananya believes that returning strategies can give some more rats and maybe the win!
Sania supports that the stealing with one piper is crucial and it should be tuned.
Jing said that adjusting the sweep tactics to reach the corners can give some 
stray rats.

Konstantin asked how the tournaments will end. By a timeout depending when a rat 
reached a gate. If much time without rat at gate, stop.
The students were asked about the outcome of the tournament
Amar believes that as we use mostly the same strategies the results will not vary much
Tingting adds that as the configuration are known, they can put effort on them.
Derek suggests that the corner case will give the upper hand to the teams.
Preetam disagrees that the results will be closed especially for the denser boards.
Robert points out that with the sweep techniques the dense will be similar and in the sparse the best thief will win.

\subsection{Monday October 5, 2015 - Presentations}

The order of the team for presenting was created from which groups are all in the
classroom at the begining of the class. The ones that all the members are here
goes first. The duration of the presentation will be 7 minutes and georgios will 
keep track of the time. A signal will be given by him when one minute is left.

First to present is group 3.
The group tried to get the 25\% of the rats and then try to get some more. They
break the problem in the number of pipers: 1, mid-piper and high-piper. This
happened because the strategy to gain some more 3\%-5\% is different when you
have 1 or more pipers. Their goals are to target the areas with more rats, areas
that are close to pipers, areas that are close to the gates and try to get some
more rats on the way to the gate. The reward is calculated by $Max R = \frac{N^2}{x+5y}$, 
where x is the distance between piper and a point P,y is the distance P and gate
 and N is the number of pipers within 10m of P.

Their individual strategy is to do a sweep when high density of rats and then 
change to low density as one piper. The high density sweep focus on going to edges
, meeting outside the gate for protection and form an arc from the gate.
Groups become competitive(stealing) when rats are sparse mainly after initial 
sweep.

Group 2:
Group 2 had the same goal of acquiring the 25\% of the rats and then try to get
as much as they could. They had created a nice representation of rats as hot 
spots in an image. With this they were sending their pipers in the darker areas
to get some rats without interference. This was their primary strategy. Other 
strategies that they used are a greedy one when all piper converge and try to
steal rats and of course the sweep strategy for the beginning.

Group 9:
This group they started doing some incremental changes to group0 code. They have
seen that going with one piper per rat has better outcome because if you go to
just one you will attract more pipers. As they had to fight the sweep strategies
of the rest they also do at first a swipe and then change to their main strategy.
Their first sweep or as they call it, teamwork strategy was to divide the pipers
into to teams. One will go far and bring some rats closer. There they will meet 
with a team that takes rats and drives them inside. This strategy was vulnerable
to stealing. 
Their main strategy is a cell one. They divide the board in cell and send the 
piper to the closest one with the most rats talking into account if there are more
pipers. So at the beginning they are doing one swipe and then go to their cell 
strategy. Their sweep strategy takes into account the sweep strategy of the other
groups (like if they go to corners). Their stealing strategy is really good thus
they have very good results in less pipers than the ones with multiple pipers.

Group 6:
Group 6 had divided their strategies into dense and sparse. They change their
strategy dynamically when the rest of the rats is less than the 25\% of the 
initial number. First they do a radial sweep in fixed angles and thy merge near
the gate to safe capture all the rats they have. Their sparse strategy is one
they call wolf pack as they are dividing the pipers into groups of 2 and send 
them near the gate and were the rats are plenty. Then they come back to the gate.
Their overall results was not good but had some individuals wins.

Group 1:
Group 1 has not implemented a sweep strategy in the whole. They created a grid
of weighted cell which in the end was similar to sweep strategy because in dense
boards were divided. The cell is targeted if it has high density with the 
following function: Cell is dense if density$> 1.3*$ avg. cell density.
The $1.3$ factor come after empirical results. Then the pipers are divided 
proportional to cell weight. If any additional piper is left then sort rats with
distance and send one piper to each one. They also experimented with k-means 
clustering. Because grid cells can cluster rats but show them as low weight cell
they started looking at a k-means approach. K-means approach was an expensive 
one so it is only good for more than 500 rats.

Group 8:
Group 8 is doing a sweep strategy when the configuration has many pipers. They 
are sweeping from the edges and towards their gate in order to get the rats in 
the edges and close to their gate. After that they change to gate stealing where
all the pipers go to one gate and try to steal rats from there. When the 
configuration is with only one piper they are splitting into to strategies. If 
the board is dense then they collect rats until the piper has $\frac{1}{5}$ of 
the initial number and then go to the gate. If sparse the piper goes to the 
closest one. The result showed that their strategies for many pipers was a good 
one.

Group 7:
The group created a metastrategy in order to dynamically change what the pipers 
will do. It calculates a clustering factor: standard deviation of piper density 
in 10-meter-wide cells. Density factor $\frac{R}{P}$: number of rats in the field 
per piper on each team. Rat-free ratio $\frac{R_f}{R}$: fraction of rats not being 
captured by any piper. If the clustering factor is high then a sweep action is
triggered. If density factor is not high then go for a closest rat strategy. Also
if density factor is high but the rat free ration is not then closest rat strategy
will be the best approach. If not then a greedy state is triggered. A potential
field function was configured in order to penalize or reward the things that are
in the field. Like add five for an enemy piper if is playing or 10 for not playing.
Also a max density algorithm is used. In order to find the best place an algorithm
with complexity of $\mathcal{O}(R^3)$ where R is the number of rats. Also they
are using a technique for getting the rats closer to the piper by toggling the
music. Their results were not the best but is good for configuration with 1 piper.

Group 4:

 





\section{Planet Builder}
\subsection{Wednesday September 30, 2015 }
Today at the beginning we talked a bit about project 1 and how it will be run and how you will get the results. A csv file will be published with the results and a script to load it in a MySQL database.

Group2 was a bit slow and they found the error and send a new source code.

Start of project 2 discussion

We presented the simulator and a document with the physics of the exercise. Most of the questions students asked can be found in the document.

Arthur asked if the velocity of the planets changes.
Yes, the velocity changes and it depends at the place of the orbit. In the document there are all the equations about this.
Robert’s question was how the new orbits will be.
The simulator helps you by having a function to calculate the new orbit. Simulator does not let you to do pushes that will result in hyperbolic orbits.
Derek asked about the energy the asteroids have in collision and in general.
In order to made the exercise simpler we assume that much of the collision energy is transformed into heat. 
Parthiban had a question about the equation of the orbits. We strongly recommended that you may want to reconstructed from the data we give but not try to solve it. It has the error of time if you want to find collisions and as it has no closed form is really difficult to solve.
Vinay: Perpendicular pushes and what is the mechanics of the push.
The pushes are the problem of the exercise. The mechanics and how the velocity changes can be found in the document.
Ananya asks about the calculations of velocity in collisions. You have to take these calculations into account in order to found the new orbit.
Cathy asked about hard pushes but the answer is that first hard pushes means that you will add too much energy. Also these pushes can lead to hyperbolic orbits.
Robert suggests that hyperbolic orbits that collide before leaving the system should be taken into account but the answer was no. As it will add unwanted complexity.
Seth asks if we the push and collision have time measurement. The answer is no. We assume that both of them are momentary. 
Sagar wanted to know what is the winning parameter. Mainly if time is taken into account. The exercise will measure the energy used. The time will just be for the time out. If you finish in the nick of time will be the same as if you finished after 10 second. 
Tingting asked how they can test will be runned and how the configuration will be tested. There will be a file input in order to have specific configuration to run.
Samar wanted to know if 2 ellipsis are intersecting will collide. The answer is that probably not. Because of the time parameter is really unlikely to be in the same place at the same time.
Artur asked about the collisions and the volume of the asteroids. The collision will only happen if the 2 orbits collide and the distance of the centers is smaller than the radii of the asteroids.
Constantin asked if it is possible to create an asteroid without velocity. In this version of the simulator is possible and it makes a linear orbit. This will change as it maybe an easy solution for the exercise.
Partiban was curious if there will be partial credit. Right now the professor had not thought about it but probably will be.
Sania and Derek asked about some the system. The orbits are all in the same direction and the simulator saves the energy we have used.
Seth asked about the if there is a way to speed up the simulation. The answer is that always you can choose to run without the gui. Also if there is not any collision in the near future the simulator has the parameter to skip such periods.
Limitation of the computation power had been arisen as a question. In general it is not but you can have it in their report/presentation.
Planets are cosmetic. No collision between them as Sagar asked.
Collisions will not lead to an asteroid leave the system. There is an exception 
Constantin said we can push all the asteroid and this will all start to collide. But as the energy is accumulated this strategy may not be the best.

And with this comment the conversation pivoted to strategies for the exercise. 
Pathiban assumes that as we have a time limit we should do multiple collision/pushes in order not to reach it. But still this will lead to many charges of energy.
Sagar suggested that the exercise has a search space and this can be traversed by an A*.
Vinay assumes that a search in time for collisions will help us not to spend energy if something will happen in the future.
Sean believes that as smaller asteroids need  fewer energy than the big ones, we should push only small ones to the big one we are creating.
Amar suggestion is we should plan for collision in further time. His strategy is that you will push one asteroid once and then you should wait to collide with the circular orbit.
Tingting adds to the A* algorithm that pushes will be different choices in the decision tree.
Kevin suggested that you set the orbits in order to make a domino effect and start pushing the inner asteroid and collide with the rest.
Preetam adds to Kevin’s strategies that will be easier to start from the outer inwards.
Vinay continues that this could build up the consumption of energy.
Jing suggested that we should find orbits that are nearly parallel and push them as they will need fewer energy.
The class suggested this can lead to timeout because you wait for the right moment.
Robert asked if we can run the simulator inside the simulator. This may help them to find position. The group0 does this.
Derek asks if the collision happens in the begin or in the end of the tick. The length of the tick will be in accordance to the radius.
Vinay suggested that we should plan colliding the outer asteroid inwards or the one that are close to the 50\% limit. 
This is fine but as the orbit changes in every collision so you must plan until the next collision.
Sania wandered if the asteroids have always the same velocity. This is not true as the velocity is depended to the radius. Asteroids are faster near the sun (perihelion) than in the the further point (aphelion). 
Seth asked if they can solve the equations of the orbits. It was already answered that is too difficult and you can not predict if there will at the same point at the same time.
Parthiban thinks that a dynamic programming solution with a bottom up approach will be a good strategy to solve the problem. But the computational time might be a drawback.
Sagar believes as there are a finite amount of states from a push to each velocity. This would be a good space to look in order to find the softer push that will lead to the closest collision.
Constantin raises again the issue of the skip time in the simulation.
Lingyan suggested that we should find the asteroids that are near push them gently to collide.
Manyi asked if they can return into circular orbit. The answer is yes but it needs specific pushes. You can look at Hohamann transfer orbit.
Sania thought that a perpendicular push will get them into higher orbit. Not this is not true due to the fact that the velocity is not always perpendicular to the radius.
Parthiban wandered if the orbit changes when asteroids pass close to each other but not collide. 
No, the asteroids does not gravity force between them.
\subsection{Wednesday October 7, 2015}
Today we run the first deliverable of each group.

Group 5:

The team at first year does some calculation and no pushes. They are doing a small push and try to find if the 2 will collide in the next 5000 days. If not collision next push.

Group 8:

They are pushing the furthest one because it is the one with the least velocity so it minimizes the energy. Then check if collides in the next 10 years and if not do some push. They are trying to collide small asteroids to one and make it big.

Group 3:

They are trying to collide asteroids in close radii. 
Derek mentions that is hard to make a push that will collide them. You have to make them have similar phase not only being close in radius.
Seth mentions that if they are heavy and having close radii is advantageous as you can push them together. 
Amar agrees with Derek as if the phases of the planets are huge then it is not easy to collide them.
Samarth suggests that a negative push (opposite to direction) will give better chances to collide if they are not in close phase.
Robert does not think that it has any advantage to push in opposite direction.
Dhruv opposes to that if anything makes them to collide faster is good.

Vinay analyses that the ellipses can intersect on 4 or 6 places. So try to find these points.
Sania believes that we can find a perfect push that we will make the asteroids to collide on the next turn. 
Constantin is concerned that as the energy is velocity squared, tiny pushes can do the job?
Vinay asks how the velocity is calculated. The velocity calculation is in the document that we provided.

Group 6:

They still keep the random selection of asteroids to push but they are limiting the pushes only for better choices.  
Artur suggests that going a bit random is good. Because the computation for doing all the analytical job is too much.
Kevin believes that we should limit the subspace of the problem in such a space that is easy to be solved analytical. 

Group 7:

Kevin says that their strategy is to push the new asteroids to circular orbits and try again.
Preetam is puzzled if this is optimal.
From the result of this configuration seems to have a competitive result.
Amar raise the issue that when you come closer to the threshold is crucial which asteroid you will push.
Nam suggests as strategy that you should start doing some random pushes and if that does not give some results try to solve it analytical
Vishal suggestion is if you just do sequential pushes you can miss easy collision that can be made.
Cathy also raises that even if we limit the space, this space may not have a solution.
Rhea ask a question if a circle is possible to have another center than the sun. No as the gravity between Sun and asteroid is the centripetal force. 

Group 3 in fast forward:
No clue that are making circular orbits

Group 2:

THe strategy of this team is similar tt group’s 7 strategy. They collide and then make the prbit circular again.
Seth asks if the simulator can print pushes and energy.
Sania asks why going again in circular orbit.
Derek says that is not optimal but it is easier to make asteroids to collide.

Group 4:

Strategy of this team is to find the 2 asteroids that are closer. Search with what push can collide them and do it. 
Cathy suggests that if the closest are near the sun can lead to huge pushes.
Amar strongly says that we have to define what closer means. By radius or phase?
It adds that if the difference is of phase then this can lead to a big push.
Sagar proposes that we can search when 2 asteroids are close and try to go backwards when would be the best time to do a push in order to collide.
Alice said that this has too many possibilities and will not always work.
Robert says that it is easy to find if they are colliding in space with the 4 points. But we also have to look if they collide also in time.
Seth suggests that we can skip times in order to see if they collide. because otherwise the computation can be huge.
The main point is that they have to solve the collision problem in sub quadratic time.


Group 9:

They do random pushes but rank the asteroids to be pushed by volume (smaller to larger ) and by distance (first the furthest).

Group 1:

They are pushing the lightest asteroid but randomly. 

We changed the configuration to have 50 asteroids and run again group 4

Preetam says that the choices for pushing now are enormous.
Parthi suggests that now going back to circular may add up to be big.
Vishal proposes that a good strategy will be to make the outer one elliptical to intersect with all of the rest.
Parthi add that this is not optimal with Vishal’s approach.
Vishal said that if we need a big push to have an eccentric orbit to intersect with all of them. So focus on makes the outer ones to collide.
Nam propose that energy and mass should be taken into consideration together. We should have the delta energy in our minds.

Deliverable for next week: You should wait many asteroids. Have a solution to the problem.

Next important thing:

Sania suggested parallel pushes but only for the asteroids that are a best fit
Vinay said we should focus on not creating 2 big asteroids because we will need big pushes at the end.
\subsection{Monday October 12, 2015}
In this course we run all the groups to see strategies.

Group 9:
Group’s strategy is to detect when 2 asteroids are closer and do the push. By close meaning also phase and distance.
Amar says that is fine for pushing always small asteroids but if we are closing to the timeout we should seek more large pushes.

Some other big asteroids were formed and those were created and Sean says that are by accident.
Preetam suggests that is nice that are created by accident because they came with no energy at all.
Robert points out that if too many will collide by accident we will not have small ones to collide.
Artur saw that some pushes are big and this is not good.
Sagar points that a mistake in a push can ruin the game.
Dhruv said that accident asteroids is good as it gives you better chances to collide with something big. 
Sania suggests that if they have an intersection with a small push maybe you can collide them.
Amar says that you can make a collision that will go near to another big one and you will need a small push to make them collide.
Sagar’s opinion is that it make too much time to calculate when and if they will collide.
Dhruv believes that big pushes are achievable if are done in perihelion or aphelion where the velocity are small. 

Group 3:
Kevin said they are trying to find angles that are best for pushing in order to collide with other asteroids. They are colliding in the outer rim.

Kevin believes they should choose the middle ring as half the mass is outside of it and they can win the game with just them and need smaller energy.
Tingting opinion is that they should converge in the middle of the half outer belt of asteroids.
Seth suggests that going for th 50% is good but in the rest maybe is an asteroid with better velocity or phase to collide with this one.
Derek points that all of this has as argument that the mass is distributed equally. It would be the best to find the ring that has 50% of the mass outside of it and use this.
Jing believes that we should focus on the denser areas of asteroids and look from there.
Vishal suggests another strategy that as all of the asteroids will end to the big one collide some together to lead them to the big will give us less energy from pushing each one to the big
Sania adds that with the collision of the 2 small can give an orbit closer to the big one and then a smaller push will be required to hit the big one.

Group 1:
Vinay analyses they are trying to minimise by searching when the velocity is smaller.
Preetam tries to implement an A* algorithm for the project.

Group 7:
Dana says that are trying to avoid accidental collisions by looking into the future.
They are focusing on the outer rims and they are looking for 40 years in the future for the minimum push.
Ananya suggests that only colliding with small ones you can add up close to 50% but then you must search if there is any asteroid that with one push will win you the game.The benefit per mass should be taken into consideration

Amar says that you can do small pushes and then some corrections


Group 4:
Dhruv told us that are finding the 5 best in the big asteroid and pick from them the one with the smallest energy to collide with the big one.

Robert said that their inner loops take too much time and they are not taking the optimal as they are limited in asteroids to avoid big calculation times.

Group 5:
They are looking in 100 days in the future to find which one is best to collide with the bigger one. But the best one (closest) may not be the best as it has big velosity ~> big push. 
Vishal points that have used inner rings which need bigger pushes thus the big energy at the end.

Dhruv suggests that threshold for pushes can be a portion of initial energy of all asteroids.
Parthi adds a suggestion that threshold can come from the pushes you have already done.
Seth’s opinion is to do  a bit more search when small numbers of pushes.
Robert adds that a threshold could be the energy from the inner asteroid to the outer one.
Samarth said that the energy of putting again the asteroids to circular orbit could be the threshold.

Group 8:

Sania said they are looking in 10 years forward to find a collision. If no collision make push to collide the 2 closest. They are doing sequential searches for pushes.

A question was asked if any is doing parallel collisions.
Derek tried but too much waste as at the end 8 big was formed and the pushes to for them was too big.
Seth adds to this as it seem not good to form many big ones.
Cathy believes that when you have 50 asteroids you can waste some parallelism.
Nam states that we have many choices and time is not threshold and even sequential search we are searching a big space of collision.
Alice asked what exactly parallel means.
It means that you are looking at the future before colliding and use the result for an upcoming push.

Group 2:
Sean tells that they are doing similar as the group 7 with return to circular. They are starting from the bigger asteroid neither from outer nor inner.

Group 6:
Tingting said that are trying to find intersection in orbits and looking in the distant future is not good. In general they are doing parallel and wasteful.

Sagar asked for state files as experiments.
\subsection{Wednesday October 14, 2015}
A conversation started on how the configuration for the tournament should be.
Cathy suggested that a time limit should be introduced. This would give better ranking for teams that did better job on the decision time. 
Professor asked if they would like a change in the direction of the asteroids. Not many students were in favor of this.
Seth suggests to add the number of pushes as a ranking option.
Preetam elaborated on how the ranking will be if they hit the time limit.
Professor said that the push and the mass that is already created will be taken into account.

After this professor asked that every team will submit a configuration that they thing is a good candidate. 

Group runs:

Group 8:
Students asked to have some asteroids really far away. Orestis changed the configuration so some of the examples have this as configuration.
Group 8 are making small clusters and then push them to the big one they are creating.

Group 4:
They are making a push every one hundred years. They are only considering small pushes and they consider only 2 angular pushes to restrict the search space of viable pushes.
Costantin asks if this reduced the search space.
Artur replied that they are trying to optimize the push and angle ratio.

Group 1:
This group’s strategy is to converge the inner asteroids to the middle and the outer to the middle. Having to search in all asteroids they can minimize the push.

Professor raises the question how the teams are dealing with the time.
Robert says they are deciding for pushes taking into account the ratio of time left and the asteroids that are in the game (eligible for push).
Kevin’s group approach is to do a deep search when the game starts to find how much a schedule will take on.
Amar said that when the time is going towards the end they allow bigger pushes to happen in order to complete the game.
Costantin only calculate how much time needs a collision to happen.

Seth suggests that a good strategy would be to try easy ones first and as the time passes you should allow bigger pushes(as Amar).
Amar makes the claim that is hard to distinguish between east and hard through time because of the phase change.
Tingting adds that if you do clustering asteroids you will have a spike at the end when you are trying to push the big ones together.
Sania suggests that a not huge push could be found in future years.

Vishal raises the question on how we determine what is huge and what is low energy.
Robert opinion is to calculate a low and huge for the configurations with the help of the move of asteroids.(how much energy needs the inner one to go to outer one- huge and vice versa).
Preetam suggests an average after some pushes.
Samarth says do search for some pushes and pick the minimum from there.
Vinay claims that if 5 asteroids (not too much) the average is not a good strategy.
Ananya tells that if 5 asteroids you can do exhaustive search. And it works as they have done it.

Group 3:
Robert tells that they are trying to get some more angles to their search space for a better collision.

Kevin says that the calculation of the center is not always the same as they are expecting.

Group 2:

Parthi describes their strategy as they have developed a function to cost each push and this helps them in which asteroid to collide.
Derek says that the current function is just for testing and they will try many of them.
The group made many collisions probably due to bugs.
 
Professor asks the class what they believe in which orbit to form their planet.
Sean believes that the best option is to collide them at the middle of the denser part.
Vishal suggests that first will have to make some collisions and then decides where is the best orbit.
Jing says that the best is to converge in the outer rim as they pushes there have lower energy.
Tingting thinks that going either from outer to inner and inner to outer is roughly the same.
Sean says the problem is not the actual end velocity but the delta between the velocities.
That means that going outer is bigger push.

Group 6:
Constantin describes the strategy as they are having their energy in a fix number and they are trying to find asteroids that will collide with such energy in near future. With the Orestis’ suggestion about how to calculate collision points can search a bigger space in less time.
Samarth asks on what angle they are trying.
Costantin replies that they are doing it randomly.



Group 7:
Kevin says that they are doing many parallel collisions in the outer rims. They are using much energy but it finishes. 
Ananya adds that with this continuous collisions are sure that they will finish the game.
Kevin also said that in average of the pushes are the same as always pushes the small ones.

Group 5:
Amar describes the strategy as an algorithm trying to find best push and low energy push between asteroids that are close to distance and phase.

Group 9:
Lingyan said that the function of the intersection lowered the search time. 
Rhea said that they are looking for the bigger asteroid in order to direct the small ones to it.
They are trying to find where the density is around the 25i\% of the total mass and then start from there. In order not to send to many asteroids they are picking only the 5 best and do the calculations. 

The deliverable for the next time is an almost ready player.

What you will deal with on your player:
Robert’s team will deal with asteroids in the same orbit.
Vinay will try to do some empirical to find some thresholds
Sania believes as they are doing good with the time to start expand a bit the search in doing some parallelisation. 
\subsection{Monday October 19, 2015}
A conversation was started regarding the configurations they submit.
Samarth asks if there will be 2 asteroids in the same orbit. (g1)
Robert suggests 2 zones of asteroids one close to sun and one far away. (g2)
Amar asks for less planets and less time.(g3)
Kevin proposal 3 zones of asteroids 47 - 6 - 47 (g7)
Jing as the default but with some asteroids far far away (g8)

After a vote g7 got the majority.

Time was also a topic in the conversation.
Preetam asks for 500 years reduce.
The majority again wants 1000 years.


Group 5 with the g7:
Sagar said that they are using the Hohmann transfer as this will guarantee them the least energy.
Amar continues that the outer orbit must be in tangential orbit and this will want the least energy.
So with that in mind they are discovering the pushes with least energy. As the time goes by the threshold of least energy push is changed in order to collide some asteroids.

Group 8
They want to take middle ring and collide with the outer ones.
Seth says that they did the right thing as more asteroids in the outer than in inner.
Vinay points out that they did some really big ones at the beginning which may not be what they wanted.
Rhea suggests that going in circular will have the least energy.
Alice said that the initial pushes was from middle to outer but this is not the issue
Dhruv says that it is a patient issue that you do not know if the first you find si the best push to do.
Parthi suggestion is to do more search in the first years to find a small one.

Group 7 
Kevin said that they are doing calculations to find the best orbit to gather the asteroids. In this configuration finds the best is in the middle layer. Also big pushes can mean that the collision is further in the future rather than immediate.
Amar has a different opinion about the amount of push and when they will collide.
Sania points that they are doing pushes early. Thus they end early with big amount of energy. They should invest more time to find better pushes.
Robert says that most of the techniques will  work with almost circular orbits so they will need less energy.
Nam points out that at the beginning all are circular. So you need a big push to disturb the system at the beginning to create randomness.
Parthi says that all seems logical. Everybody should implement their idea and will see in the tournament how well we did.

Group 1.
They are getting everything to the heaviest thus the pushed everything to the inner (samarth).

A conversation on what config will be used
Sania would like to see some time variation.
Cathy is not in favor of this as the time is a main option
Preetam wants the config to be known earlier to run some experiments
Kevin would like physical realism to the configuration. Meaning not all of the asteroids to one orbit.
Sagar points out that this can be done because of phase difference.

We have reached a conclusion of using until 100 asteroids.
Jing asked how the result will be stated if they do not create the planet.
Mostly on the energy used.
Ananya asks again about minimum time.
She got a reassurance that will not be evil with too many asteroids in small time frame.

Group 2
Pushed middle asteroids to inner.

Group 1
They did not finish because the push wanted to end the game was above the threshold.

Group 4
Dhruv said that their changed their strategy to Hohmann transfer. They do not have any threshold so all the pushes are viable.

Costa asked the team if the elliptic orbits that are visible are remnants of not correctly calculated collisions.
Sagar answered no.

Group 9
They were doing too much thinking. They have never seen so much before. Probably because of the new configuration
They are trying to find the mean best push and looking also into the future collisions
They are printing the prediction and they are using elliptic orbits so it is harder

Group 7 
Also too much thinking. Manyi says that as they are having a tight budget they collide the inner as they have better opportunities to collide.
Tingting assumes that if the time is limit is small then the far away ones can not be part of the solution as it needs more time to compute new orbits.

Group 2

Parti says that they are having a function for finding the best push to go from outer to inner.
The one that have is not the best right now.
Derek says that they are using dynamic programming to find best push for new orbits.

The student asked to make some predictions of which strategy would be the winning.
Preetam believes that the time limit is the point we should focus.
Cathy believes that the strategy that takes time into account will be the best
Vinat that finding the least first push is the turning point.
Sania adds that groups that targeting efficiency will not be good with time limit.
\subsection{Monday October 26, 2015 - Presentations}

Group 5:

They implement a number of strategies. First was the iterative approach. But the
number of computations per day were too much. Then they implemented the gradient
descent algorithm. The complexity now is lower in logarithmic time. Then an 
variation of Hohmann transfer was coded. As Hohmann transfer is only for objects
in circular orbits it is not good for ellipsoid orbits. So they are calculating 
the Hohmann transfer energy and used it as a limit. The picking of the asteroids
were done by calculating the N asteroids with the lowest Hohmann transfer energies
that make 50\% of the mass. In order to have limit the energy in general they 
calculated the Hohmann energy needed from the smallest to the largest orbit and
used in a multiplier: $\frac{\frac{\# of remaining asteroids}{initial \# of asteroids}}{(\frac{time remaining}{time limit})^2}*\frac{avgPush Energy}{Maximum Hohmann Energy}$.
They also restricted the number of angles for pushing in exhaustive search and
multiple asteroids in same orbit. Asteroids in same orbit was handled by pushing 
the lightest to change velocity and collide with the other. Also the sink asteroid 
is the one with the lowest average Hohmann transfer energy.

Group 7:
This group dominated the tournament. They choose to have only circular orbits. 
After every collision they do small pushes to make the orbit circular. This strategy
helped them because circular orbits can described analytically easier than the
ellipsoid one. It easier to make collision happen as the formula is easy to 
formulate. Only 2 games were stopped because of consumption of the CPU time.

Group 9:
Group 9 starts by finding the sink asteroid finding the smaller distance with lowest
weight amongst the asteroids. Then searching for the largest distance/mass ratio.
In order to prune the search space they only look for pushes parallel to velocity.
With this options they only have 3 situations that an asteroid has. Inner than
the sink asteroid. Outer and in an intersecting orbit. Even though they did not 
win in any match they had good results.

Group 3:
Group 3 only uses the half of Hohmann transfer in order to collide 2 asteroids.
They do not do a second push as this kind of transfer is creating almost circular
orbits. With such approach they always find the lowest sum of partial Hohmann
transfer between all asteroids in order to choose the best ones that can create
a 50\% mass asteroid. Their apprach gave them the most 3rd places amongst the groups.

Group 4:
Even though this group tried not to use Hohmann transfer, after the not good 
results they changed their strategy also to Hohmann. They almost what other teams
have done. Specify a sink and then prune the search tree by looking at less angles
and less energy in pushes. They pushed asteroid in tangent orbits of the sink 
asteroid's orbit. Still had not the best results. Performed poorly in dense 
configurations . But they dominated the sparse ones.

Group 2:
Group 2 tried to have many heuristics and had a goal to finish all the games.
If the time was about to end then many pushes were done in order to finish the 
game. Only asteroids in same orbit as the sink are only pushed away. They were
stable with dominating 3rd and 4th place. Also with their approach they always
finish.

Group 1:
Group 1 are choosing the sink in the zone between 60\% and 75\% of the accumulated 
mass of the asteroids. They limiting the lookup time when the time is coming to 
the end. Their main strategy is again Hohmann transfer to the sink asteroid. The
had 60\% of the times in 2nd or 3rd place.

Group 6:
Group's goals were to finish the game, avoid big pushes and do the best pick for
a destination asteroid. They choose the destination asteroid in the densest part
of the map in the hope that this local optimum will lead to an efficient solution.
They have implemented a function type of $x^3$ of limiting the push energy with
time. Because when the end is near you have to let bigger pushes in order to finish.
This strategy did not result in the best results as the function that had picked
let the limit too high too soon.

Group 8:
Group 8 implemented a clustering algorithm in order to push the asteroids that 
are closer to each other. This idea believed that will get them with the lowest
energy push. The pushes were also done by the Hohmann transfer. Specifically two
variations an analyzed and reverse Hohmann transfer. At the end they had an issue
of convergence and that cost them a big portion of energy.

\section{Cookie Cutter II}
\subsection{Wednesday October 21, 2015}
First some questions were asked from the TAs about the randomness of their planet builder.
In order not to run many experiments with same configurations we will do pseudo randomness with a specific seed in Random. 
Also the students said that using maps from today will not be fair as if it is picked the particular team will have an advantage.

Problem 3 Cookie cutter

Students were asking what happens when illegal move is detected.
The simulator throws exception.
Cathy asked which size and what cutter a player can use.
Except the first turn the user can choose whatever he wants.
Rhea asked if they can have holes in the cutters. Yes you can
Seth asked about how the cutters can be placed. - No mirror only spin around.
Kevin asked about what metrics should we care. Dough covered or win - Win
Preetam asked a strategy question about U shaped cutters but was get on hold.
Tingting wandered if they have any limitations on how many times they can use a cutter. No limit
Kevin wanted to know if they will be cpu limit. Probably yes.

Then we did the group assignment 

Strategy questions:

Artur said that choosing shape is important. Your pieces must give advantage to your pieces and blocking your opponent.
Tingting asked what we mean by helpful. It means that your 8 piece can be helpful to your 11 piece (creating a weird shape that only your pieces can fit)
Kevin suggested that the 8 piece cannot block the 11 piece.
Vinay’s opinion is to create pieces that can fit together
Namar said that you big pieces can create space for the small one to fit.
Amar thinks that there are 2 main strategies. One that ignores opponent movement until a point and one that competes from start
Dhruv believes that making gaps is the winning point as you can always fill it near the end.
Sagar states that the pieces should mend together.
Preetam suggests the usage of the 11 piece more as it is more greedy to get dough.
Sania states that if you figure out the opponent strategy you should start be adversarial.
Cathy strategy is to try to fill the board until a limit. There are many options in the beginning so go random at the beginning and then adversarial
Parthi believes that you should go for the edges at the beginning because at the end will be difficult to fit.
Jing is against parthi strategy because the corners can be easily blocked by one piece of the opponent.
Sean believes that blocking the 11 piece of the adversary is the winning strategy
Robert believes an L level cutter and with a cascading strategy can block the opponent.
Vishal point that an S shape will be even better.
Alice states that if one chooses L and the opponent S then you are in a worse position
Derek point of view is that if the opponent chooses L choose cube in the smaller piece
Nam states that cube can be attacked easily because of the shape.
Ananya believes that this is not one way analysis of the cube. Cube can ruin also L shape cutters
Rhea said that using irregular cutters can leave more dough unusable as no one regular piece will fit in those.
Preetam suggests that you should create dead space for you opponent.
Seth agrees that as you know what shapes he has you can create dead space he cannot cover
Lingyuam suggests that the cutter should be horizontal in order to subdivide better
Costa believes that you can always defeat a strategy of covering pieces by placing next to the opponent piece to block him
Tingting adds that there is a flexibility on where you can block as you go all directions
Manyi suggests that the 11 and 8 piece combined can create a hole that fits your 5 piece

Diana believes that you 5 piece should interfere with the hole of the 11 piece of your opponent
Nam said that the choice of the 11 piece is crucial as it it the one that covers most of the dough
If you can create a U shape you can interfere more.
Manyi suggests that the strategy can find what shapes you can create and attack them
Preetam believes that the 11 and 5 pieces must be constructive.
Vinay suggestion is to find the complementary piece of your opponent
Meaning that it should fit the 11 piece
Kevin suggests that having an irregular 11 piece can not stuck perfectly so a line piece is better.
\subsection{Wednesday October 28, 2015}
First some questions were asked from the TAs about the randomness of their planet builder.
In order not to run many experiments with same configurations we will do pseudo randomness with a specific seed in Random. 
Also the students said that using maps from today will not be fair as if it is picked the particular team will have an advantage.

Problem 3 Cookie cutter

Students were asking what happens when illegal move is detected.
The simulator throws exception.
Cathy asked which size and what cutter a player can use.
Except the first turn the user can choose whatever he wants.
Rhea asked if they can have holes in the cutters. Yes you can
Seth asked about how the cutters can be placed. - No mirror only spin around.
Kevin asked about what metrics should we care. Dough covered or win - Win
Preetam asked a strategy question about U shaped cutters but was get on hold.
Tingting wandered if they have any limitations on how many times they can use a cutter. No limit
Kevin wanted to know if they will be cpu limit. Probably yes.

Then we did the group assignment 

Strategy questions:

Artur said that choosing shape is important. Your pieces must give advantage to your pieces and blocking your opponent.
Tingting asked what we mean by helpful. It means that your 8 piece can be helpful to your 11 piece (creating a weird shape that only your pieces can fit)
Kevin suggested that the 8 piece cannot block the 11 piece.
Vinay’s opinion is to create pieces that can fit together
Namar said that you big pieces can create space for the small one to fit.
Amar thinks that there are 2 main strategies. One that ignores opponent movement until a point and one that competes from start
Dhruv believes that making gaps is the winning point as you can always fill it near the end.
Sagar states that the pieces should mend together.
Preetam suggests the usage of the 11 piece more as it is more greedy to get dough.
Sania states that if you figure out the opponent strategy you should start be adversarial.
Cathy strategy is to try to fill the board until a limit. There are many options in the beginning so go random at the beginning and then adversarial
Parthi believes that you should go for the edges at the beginning because at the end will be difficult to fit.
Jing is against parthi strategy because the corners can be easily blocked by one piece of the opponent.
Sean believes that blocking the 11 piece of the adversary is the winning strategy
Robert believes an L level cutter and with a cascading strategy can block the opponent.
Vishal point that an S shape will be even better.
Alice states that if one chooses L and the opponent S then you are in a worse position
Derek point of view is that if the opponent chooses L choose cube in the smaller piece
Nam states that cube can be attacked easily because of the shape.
Ananya believes that this is not one way analysis of the cube. Cube can ruin also L shape cutters
Rhea said that using irregular cutters can leave more dough unusable as no one regular piece will fit in those.
Preetam suggests that you should create dead space for you opponent.
Seth agrees that as you know what shapes he has you can create dead space he cannot cover
Lingyuam suggests that the cutter should be horizontal in order to subdivide better
Costa believes that you can always defeat a strategy of covering pieces by placing next to the opponent piece to block him
Tingting adds that there is a flexibility on where you can block as you go all directions
Manyi suggests that the 11 and 8 piece combined can create a hole that fits your 5 piece

Diana believes that you 5 piece should interfere with the hole of the 11 piece of your opponent
Nam said that the choice of the 11 piece is crucial as it it the one that covers most of the dough
If you can create a U shape you can interfere more.
Manyi suggests that the strategy can find what shapes you can create and attack them
Preetam believes that the 11 and 5 pieces must be constructive.
Vinay suggestion is to find the complementary piece of your opponent
Meaning that it should fit the 11 piece
Kevin suggests that having an irregular 11 piece can not stuck perfectly so a line piece is better.
\subsection{Wednesday November 4, 2015}
Group 1-4
They choose similar shapes and it was awkward.
The players did not change much from the previous time.
Robert asked if the teams had a backup strategy if a piece selection failed. Because most of them they did not in the previous.
Preetam replied that at least them they are going with exhaustive search in the pieces they want.

Group 1 strategy tries to cover the board starting from the server and go circular around the center.
Group 4 goes left to right from the top left corner.
As the groups did not interfere the professor asked if it si good or bad to not interfere.
Seth replied that the group 1 lost because they mend 11 pieces together and do not let space for the 5 piece.
Tingting pointed out that utilizing the edges are creating space only for them.
Lingyuan answering the professor’s question said that is good to interfere but you may miss some opportunities while being focused only to interfere. 
Alice realises that the outcome would have been close to tie if they would not have mend the 11 piece together.
Manyi believes that if the other team is not interfering then you should not either.
Kevin says that at the end platin more the 11 piece is the goal. As these will get you more dough and probably create space for the smaller pieces.
Amar points that if we choose the opponents smaller piece then it will either work for us or it will not get the 100% of the mending.
Derek points out that placing toy piece in the center is more vulnerable to an opponent piece.
Sagar believes that the group 4 is not easily interfered.  Meaning as they go for edges their 11 piece is obtaining 16 piece dough as the opponent cannot interfere with the space inside.
Continuing Sagar suggests that with a straight line near the edge you are going for 22 the maximum for such piece.
Robert disagrees that you cannot interfere. You can if you go 1 or 2 lines from the edge.
Arthur also suggested that an irregular shape can help interfere. 


2 vs 7
Both went for straight line. At the end they got opposite L pieces.
Sagar has demonstrated how is possible to create an L shape corner only for them.
Nam analyzed that with the straight line can break the opponent piece but as they also had similar strategy they have problem.
Samarth also said that their strategy is to count how many moves of your opponent you destroy by your move.
Amar added that you should count also how many of your moves you increase or slightly decrease.


5 vs 9
Group 5 are using diagonal pieces.
Amar suggests that this piece creates space only their piece can go.
Seth points that their strategy is to have 8 and 5 pieces to mend with how the 11 piece is used.
Rhea points out that if the opponent is adversarial the strategy of the 9th group would be in trouble.
Cathy thinks that as the 5 group is using diagonal they create space that can be used only by them.
Ananya said that the 5 diagonal piece can fit to many places that are created in random.
Vishal points out that group 9 strategy does not create a closed space only for them.
Liuyang  said that if you piece is interrupting many diagonals then you are interrupting the groups 5 strategy.


6 vs 3
Kevin states that their strategy is to interrupt the convex hull of the opponent 11 piece.
Amar points that if opponent randomizes their moves then stacking can help.
Sagar asked what happens if the straight line cannot be chosen. THe answer is that are going for almost straight.


6 vs 2
Group 2 is choosing straight lines to cover most of the board.
Sania stated that group 6 went well even everybody thought otherwise
Amar believes that the problem was the 8 piece as straight line which was not used as much.

A question was asked about what is board control.
Sagar answered that is to create holes only for you to use in the future.
Costa stated that board control should derive from your strategy.
Robert suggested that it will be nice to see 2 groups that are trying to control the board.


5 vs 6
Robert points out that their strategy also can understand if an area has space only for them to fit not to play  it but go and interfere somewhere else.



deliverable: near final strategy.
Important things that should be dealt with:
Seth said that they should play with other players and find out where they fail.
Cathy suggested that should spot weakness and either attack them if in opponent or fix them.
Ananya believes a selection of shapes should have a bit more thought
Jing believes that the interfering should be dynamic and not straight forward.
\subsection{Monday November 9, 2015}
Thursday midnight will be the deadline for the cookie cutter project


6 vs 8
Straight pieces vs irregular
Group 8 are trying to be destructive by placing their piece near the opponent.
Vinay pointed out that they played early their 8 piece in order to block potential
move of the opponent.
Dhruv believes that the destructive strategy was the error that made the group 8 to lose.
Rhea points out that the irregular pieces can win the convex hull strategy
Sagar opinion is that placing a piece near the opponent’s is not the best destructive strategy
Alice said that this strategy is only good for the stacking strategy


1 vs 7
Lingyuan addresses the strategy that they are using: every cell has a possible number of moves that can be participate. So make this number minimal for opponent.
Sagar added that picking the 5 piece as 3x3 convex hull left less moves with 2 pieces
Costa thought that leaving 2 pieces can mean that the 11 could play.
Ananya pointed out that they did not at the end because they have placed them correctly
Robert believed that the destructive strategy is easier than constructive.
Preetam said that the point is not to be destructive to yourself.
Vishal added that tiling is not aggressive.


2 vs 4
straight vs piece with hole.
Seth believes that they lost as they did not place the piece near to edges in order to block the straight piece.
Sania added that the placing of destructive piece was not done properly leaving enough space for the opponent piece.
Cathy points out that the 8 piece of g2 was not placed at all.
Tingting added that the g4 destroyed the inside of their 11 point on purpose.
Artur said that this was the best move at this point to eliminate moves from the opponent
Nam suggested that blocking straight line horizontally can have better effects.
Costa adds that it is not horizontal or vertical but how the opponent is playing.
Vinay added that it is essential for this pieces with holes to prevent stacking.


5 vs 3
Dhruv points out that g5 left too much space between their pieces for the 11 piece of g3 to be played.
Ananya answered that they are trying to spread out to control the board.

A question have arisen if spreading out is good for the board control?
Robert answered that spreading with one piece is not good as it leaves too much space for the opponent.
Alice asked if the neighbor function returns if out of bounds. Orestis replied that it is not.

9 vs 6 
Parthi believed that L pieces can be a counter piece for straight lines
Orestis’ opinion is that diagonal is better on that.
Sagar adds that blocking line piece is hard.

Robert suggested to see 2 line strategies.


3 vs 6

Kevin believed that the end will decide who will win.
Parthi added that at the end you can have deeper search.
Alice believed that it would be better to limit your space in the space with more free space and do there a deep search.
Sagar was negative to this because at that time you may not be able to turn the tide i you r favor.

\subsection{Wednesday November 11, 2015}
Today there were some suggestions of what game we should see instead of randomly choise.

Dhruv suggested a game between diagonal and line
Sean wanted a game with line vs line to see fallback
Robert asked to see a  time constraint game as they were not sure what it will happen.


2 vs 3 

both go for linear, 2 go for diagonal 3 to hockey stick

Sagar said that they space they left was due to not create enough space for the opponent’s piece.
Tingting asked if they have done an exhaustive search.
Sagar replied that they did not. But still they left movements that can only played by them for the end.
Lingyan added that theirs 5 piece blocked some moves of the opponents thus it was palyed a bit early.
Ananya believed that this strategy is good because you are leaving space that no one can use and blocks the opponent.
Artur suggested that this is not blocking because you block one move. They may have 3 or 4.
Dhruv suggested that using so early the 5 you will lose much dough.
Vinay suggested that you should play the 8 piece to lose less.


From the discussion found that many teams were trying to create antistrategies for specific teams.
Professor asked if it is better if you know with whom you are against.
Preetam said that is much better.
Robert added that it is possible to have better result knowing your opponent.
Sagar added that by knowing the players they identified which strategies can win and create counter strategy to beat them.
Seth believes that it is fair as all can do it.
Derek suggested they might change the strategy in the next few hours.
Costa added that they will do whatever to win.
Professor said that this not exactly how the course is. It is build for collaboration between teams
Vinay prefer to detect a specific piece and build a counter strategy for this then it is fair.
Seth also added that you only know what the last deliverable and not what in the final will be.


8 vs 4
irregular vs stacking
stacking win but too much space not used because the 5 piece was blocked out.


8 vs 1
Robert adds that if you choose a 11 diagonal piece you should choose the smaller to be similar.
Sagar said try to understand you opponent. You should choose your 8 piece to be constructive with you and destructive for your opponent.
Dhruv added that in destructive strategies you have more free dough in the end.

8 vs 2
group 2 has some space left for the 11 at the end


5 vs 7
no destructive strategy
Artur said that is risky letting your opponent do whatever he likes.


7 vs 6
straight line tvs corner.

Preetam believed that there will be many closes matches
Sania added players will change a lot so players that do player checking may lose.
Nam believes that straight lines be the winner strategy.
Vinay straight lines was always a bit better.

Professor asked if there is any advantage to start first.
Sagar said  that it is not an advantage as you use your 5 piece instead of 11 (6 pieces less)
Seth added  that the one who put his 11 piece first will impose his will.

How to put the 5 piece in first turn.
Costa proposed not in the corner but somewhere to be destructive.
Nam randomly will not help you but also may help the opponent.
Tingting believed that a strategy, destructive or constructive, will decide where to place it.
\subsection{Monday November 16, 2015}
As in the previous presentations where 9 teams presented, the time of each group
will be 6 minutes. George will give you a signal when one minute is left.

Group 9:
Group 9 gave an overview of most of the strategies. Also for them the selection
of shapes goes as: Lines -> hokey shape -> L shape families. They use multiple 
strategies over time. Like they are looking if they can counter a team that has
an 5 piece that will fit a gap in the 11 piece. Their main strategy is a defensive
one which they create queues of L shapes through the center of the board. That
helps them to take control of the board and has a secondary strategy (save-for-later)
as they are creating space that only their 11 piece can fit. They also explain
two other constructive strategies. A square named one, that creates squares but
leave too much space open. An aggressive one that create queues of 11 piece shapes
with the same drawback as the previous one. They did not do as well as other teams
in overall victories but the play well against constructive teams but had the 
opposite for the destructive ones. Also the counter piece works good.

Group 5:
The team started with alphabetic shapes but they understood that using diagonal
pieces had better results. This strategy is constructive for the team but destructive
for the rest. In choosing pieces they always start from straight line in order to
block their opponent. They created a cost function of 

$cost = distanceFromCenterOfMass * 
\frac{\sum_{pieces}{sizeOfPiece*numberOfOurMovesOfSizeDestroyed}}{\sum_{pieces}{sizeOfPiece*numberOfOpponentMovesOfSizeDestroyed}}$

In the corner case of $numberOfOpponentMovesOfSizeDestroyed = 0$ then the cost 
is an value of $Integer.MAX_VALUE$. So they choose the move with the smallest cost
value. If all of moves with the 11 piece have $Integer.MAX_VALUE$ cost then these
moves can only be filled by them. Then they turn to he 8 piece. If all moves are
$Integer.MAX_VALUE$ then all of the remaining moves cannot affect the opponent.
So do whichever one is more constructive. They won many times against groups 
1,2,4,8,9 and lost against the rest. Against group 3 they were devastated as the
average score difference was 186 points. 

Group 2
Their strategy is to lookahead in depth 2. For every empty square they calculated
a score. The scoring function is to  

Their backup strategy is to choose diagonals or hooks. Both strategies are not 
optimized and this is depicted to the results. They placed 5th overall with
catholic lose to group 9 who had won 16 games out of 80. But they had the 2nd
average score per game.

Group 8:
The team's choice is to use irregular shape. They are looking on 1 depth search
for best placement. They also using a purely destructive placement. They also 
try to counter the 11 piece of the opponent. The placement is done by calculating
the centroid and place their piece near them. If they cannot find one then put 
one in the neighborhood of the last piece. Their scoring function for choosing 
the best place is $score_{dough,shapes}(m) = w_{11}n_{11} + w_8n_8 + w_5n_5$ and
they choose the best by finding the max of $score_{dough,myShapes}(m) - 
score_{dough,opponentShapes}(m)$. For their results they won most games against
groups 1,4,9. Also they created a really good destructive player as it created
much dead space and so the score was forced to be low.

Group 3:
Group 3 wanted to design a scoring function that will evaluate a given board.
But a generic search method to play a sequence of moves is too costly. So they 
choose to play the opening moves with a more loosely strategy and when the beard
has less moves then go their core strategy. For their function they wanted to be 
claiming space from their opponent and less for their pieces. Also space that 
cannot be claimed from the opponent should not be considered and favor larger 
pieces. Another main issue is to take into account interference between possible
moves. 
$\Delta f(C) = \sum\limits_{(i,j)\notin C} (X_{i,j}' - X_{i,j}) + \sum\limits_{(i,j)} (Y_{i,j}' - Y_{i,j})$ 
where $X_{i,j} \in \{0,5,8,11\}$ the size of the cutter. They had the best results
as they won 76 out of 80. They only lost sometimes to group 7.

Group 1:
The group did some experiments on its own and created some strategies that was
specific to counter act the strategies of specific players. The strategies that
was created was: top down tiling with U shapes, center to edge tiling, min max 
search and optimization. All this three are used for specific teams. For the 
selection of the shape for the 2 first they are selecting a U shape 11 piece with
a hole. For the min max they chose a line. In their experiments for 16 games they
won the 11. Two times group 4 and 7 and one the group 5. But on average they had
one of the lowest scores. The team started an argument about groups that are not 
doing much until the last deliverable. This is counter productive and in contrast
on what the course should be like.

Group 4:
Team decided to chose pieces that can fit together. They are trying to place their 
11 piece in such a way that only the 5 piece can fit. For their 8 piece placement 
they had a function in order to find which position is blocking more the opponent
move. They did not perform well except when playing with groups 1 and 9 (10 and 6
wins respectively). They mention that lines was the panacea of this project. Also
understood that their strategy is too feeble to perform well against all teams.

Group 6:
Group 6 is choosing from line, diagonal and L shape their 11 piece. The rest 
pieces are selected to fit the 11 piece. Their strategy is agnostic of what piece
they will select. They are placing pieces that way in order to minimize opponents 
moves. Their function to minimize is 
\begin{equation*}
s = \sum\limits_mWNBLOCK_{self}(m,\frac{1}{2}) - WNBLOCK_{opponent}(m,1) - \lvert m\rvert
\end{equation*}

where m is is a move under consideration, $\lvert m\rvert$ is the size of the shape placed in 
move m, and WNBLOCK is defined as

\begin{equation*}
WNBLOCK_k(m,p) = \sum\limits_{m \in BLOCKED_k} \lvert m\rvert^P
\end{equation*}

and BLOCKED represents the set of moves blocked by m. No general strategy that 
can be win all other teams by being only constructive or destructive. Most were
good against certain types of polyonimos, so strategies cound be tuned to specific 
teams. They score particular well and was 4th on the numbers of wins. 

Group 7:
Group 7 started with a problem ana;ysis of which shapes can be used and where to 
position the pieces. Then for their strategy they made the 8 piece to fit with 
the 11 piece. This gave them a more space to claim than making the 5 piece to fit.
They were stacking their 11 piece in order only their 8 piece to fit. For being
destructive they were bocking where the 11 piece of their opponent can be placed.
This strategy make them rank in the 2nd place with the most wins. Only from group
3 had lost more than 5 times.

\section{Work the Room}
\subsection{Wednesday November 18, 2015}
First discussion about project 4.

Rules:

Amar asked if everybody moves? Yes each guest is a player. So everybody moves to 
collect wisdom.
Manyi wandered how win is achieved? Try to accommodate more wisdom.
Kevin asked how this will be calculated. With the average of every player you 
have in the game.
Diana asked if the players from each team are also friends. Yes
Seth asked if you can stalk someone. Yes it is possible but probably you will 
lose opportunities.
Preetam had stated the the win could be up to each team as it may vary because 
of different strategies.
Robert wanted to know if the wisdom numbers is the same for each player. Yes each 
game has random {0, 10 ,20} values but each player has the same.
Sagar asked when a player is free to start a conversation. After the end of one 
you have one move and then you can start a conversation.
Manyi asked if can stay still. Yes you can.
Nam asked if the number of soulmates varies as the number of people in the party. 
No one soulmate even if 1000 people.
Derek asked if you know your friends. You are figuring out by the wisdom of the 
player. You cannot communicate between your instances (no static values). 
Sania wandered if soulmates are reflective. Yes!
Amar asked about the initial positions. They are random!
Lingyan asked how you are making a decision of whom to go and how you know your 
friends.
Each player has unique id and you can see everyone that is in 6 meters from you.
Seth asked what happens when people are in equal distance. Too much noise in 
conversation you do not gain any wisdom. 
Tingting asked how many friends you have. It is an argument.
Sagar asked what happens when more than one person is at 0.5 meters. This is a 
strategy question.
Nam pointed that it will be better to take some time to walk the room and not to 
wait for specific person to end a conversation (stalking option).
Costa asked about moving. It is a teleporting ability. You have one free move 
after conversation ends. Otherwise you can be stopped from someone. 
Sagar asked for friends of a friends. No, they are not also your friends.
Amar asked about more things to visualize. Up to teams to persuade Orestis.
Amar asked if there will be any scaling in time and participants. Maybe something 
like more participants less time.

Strategy questions:



Preetam points that as no scoring metric is in place the strategy will be different 
for each player.\\
Yes but you always should go for the best.\\
Tingting said that you should position yourself at a corner in order not to be interrupted.
Vinay believes that you should minimize the time not talking. You should go greedy.
Robert adds that going greedy might be good as it gives better radius in walking the room.
Dhruv believes that it is crucial when you will leave a conversation in order not 
to be interfered.
Cathy believes that the strategy should change with how many people are in the 
room. More players then use up the time but if less try to expand in the room.
Derek asks when a conversation ends? When another person is coming nearer? No it 
stops when one or both wants to stop.
Rhea believes that we should always left some time with each person.
Vishal believes that we should have history of the conversations. Then change 
strategy on how much you have already talked.
Vinay asked if number of different people talked is a viable win strategy. Not 
how the project was build.
Sagar added that when talk to strangers smaller threshold will prevail in conversations
Also trying to find your soulmate will give you an estimation what is happening in the group.

Deliverable for next time: initial player, Run with small numbers

In what you will focus:
Seth will work on setting up infrastructure and keep minutes of conversation
Sania how much information want to give to others
Jing choose crowded place might have not the best result.
Sagar said if something is illegal you stay put. 
Ananya, if you know how much friends you have you can base your strategy on this.

\subsection{Monday November 23, 2015}
The conversation started on how we should play the game.

The main idea is not to have many guests for now. Like 2 groups with 9 guests each or 3 groups with 6 guests.

Robert asked that if we fill the room with same players (same team) we will not be good as all we will have the same strategy.
Sean asked if for small groups the time can be smaller. The time is fixed as a party cannot be less than 3 hours.
Sagar asked what the players should maximize. Because there will be too many players with which they will compete.

A poll for how many players would compete in our experiment gave that most of the people wanted many players competing. So all was used.
Costa pointed out that the gui is not helpful on understanding what it is going on.
Robert asked if it will be nice to add how many points a player can gather. This will give an insight on what is happening.

We talked about how many friends we wanted for each player.
We took a pole and the result was that we should use 4.

Amar points out that if there are 2 teams that did not respect the threshold will be bad as conversation will stop abruptly.
Dhruv added that players that are stalking others will lose turns with such strategy
Robert believes that even if you find your soulmate, you can be interrupted and at the end to lose your soul mate for the rest of the party.
Vishal point of view is, if you regularly schedule for new conversation mate you can find it again.
Jing said that they have implemented a threshold. If you get interrupted 5 times then you stop conversation.
Parthi believes that it is not critical as you can talk also to other players.
Sagar suggests that a regular move between the players can solve this problem. Move north both parties of conversation to move away from the interruptor. Also when a player gather all wisdom can start with such behaviour to interrupt conversations.
Derek asked for some explanation of the conversation mechanism. How the wisdom I gathered when interrupted and what happens with the interrupter. 
Tingting added that the easier would be to get closer to the person you were talking.
Seth points that if both do the same move then may move away from each other.
Amar believes that going to the wall will have less probability to get interrupted.
Parthi added 2 points in the conversation.First you should implement a promise. Like a place to go if interrupted. Second in general case the interrupter will go away as it will just pass by.
Sagar suggestion is to have a function for the remaining wisdom to create a place to go. Malicious interrupters may always be near. 
Vinay added that such function is not in the spirit of the game.
Costa also added that as the game does leave us any communication should we stick with such function.
Artur disagrees as we said that no communication should be happen.
Sania believes that as this is for interrupting it should be fine.

group8
Robert said that they are going first for the strangers and then for friends. He believes that you can estimate what is best when the simulation starts. Sometimes going for your soulmate is not the best.

group3
Amar said that their strategy is going for wisdom. If you find someone that gives more go for him.
Artur adds that going for soulmate was not productive.

group2
Dhruv is going for similar strategy but they found out that there are many errors when you were trying to go close to a person you want start talking to. So they are going $\frac{1}{3}$ of the way.Hoping that this will cover the distance faster.

Robert added that going 0.5 and going on with this at the end it will converge.
Amar believes that if you are not in the minimum distance you should move closer.
Cathy points out that for gathering information you do not want to be close (in 6 meter you can see)
Derek added that if the person does not want to talk to you. So if he stays put then you lose a step if you optimize.


group5
The group is trying to gather information and try for the person that probably wants to talk with them with more wisdom.

Diana continues saying that players without many players around them may go away.
Sania added that people maybe are talking but you cannot see it
Jing’s opinion was that if you interrupt a soulmate conversation you will lose time as they will want to continue talking
Alice proposed to make clusters and go tot the middle of them for better chances.
Amar disagrees saying that cluster density will not suffice as the room is sparse.
 
group6
Manyi analyzes the strategy to be similar with group5. They gather information by “saying hello” to everybody.
Kevin added that this can lead to problems if a team has programmed to unfavor such social players.

group4
Diana analyzed their strategy as more destructive as it tries to get in the middle of conversations to take some wisdom and make others lose.

group1
Sean explained that they first look for friends and then pick up strangers

Orestis pointed out that you should optimize  that for every turn you talk with someone. It does not matter what he is.

Sania believes that you should always take all wisdom from friends
Vinay added that you should go for friends that wants to talk to you.

A question have arisen if you should remember things and what these are.
Robert said that you should not remember position as it will be pointless. You should remember whom you have seen before in order to calculate how much wisdom you still have in the game
Sania believes that will be helpful to remember how the conversation ended in order not to lose time with a guy does not want to talk to you.

group9
Seth teams is trying to select the next conversation while you are at the end of a conversation.

How you approach to do conversation:
Dhruv opinion is that you should go to $0.5$ in order not to get interrupted 
Derek believes that you should go for around 2 meters in order to start.

group7
Costa said that are prioritizing amongst soulmate and friends an go greeedy
\subsection{Wednesday November 25, 2015}
The course started again with a conversation on how the experiments will be run.
 

Professor asked if going for experiments with 2 teams instead of many.
 

Sania believes that going with many teams will not give any good implication.
Dhruv added that with 2 teams would be better to see the behaviour of the player.
Amar was suggesting that with only one team the result will be different.
Artur was disagrreing with this notion as it should not be completely.

g8 vs g1

g8
Robert said that their previous player was not doing the thinge they described previously. From this version they were looking for player conversations.

g1
Vishal explained that their player is remembering people with whom they have talked and how the conversation ended. They add ignore list for people left conversation as potential they do not have more wisdom to gain so they will not speak them again.

Professor asked if this is a good strategy.

Ananya said that is good as the wisdom of this person to the other is up. So trying to talk to him to get more wisdom will be stalking.
Sean also believes that is a good strategy. Breaking a conversation maybe a hint of adversarial tactique. So stop talking for a while to this person.
Cathy added that is good at the beginning as you have many people to look for wisdom. But when the end is near you may want that wisdom.
Pathi went a bit more saying that you should see how much you took. If it is near ten then the wisdom is up. if not then maybe you can have some more.
Robert argued that with not stable distribution is hard to know if you get the whole wisdom or not. So trying to reach maximum is not viable as you do not have same wisdom per person.
Nam suggested that the problem resembles prisoner’s dilemma. So it is better to be cooperative.
Preetam disagreed as it is one dilemma per run and it is cannot be related.
Professor responded that if each conversation is taken as a dilemma at the end you have the iterative part of the dilemma.
Kevin added that it does not resembles prisoner's dilemma. Because we cannot be sure that the cumulative result is advantageous for both. It seems as a chicken out problem. The one that stays more, is penalized more.
Amar believes in an altruistic society that you keep talking to the other to also have the same wisdom.
Robert replied to that with if more wisdom on the game then you are not maximizing yours.
g7 vs g6
g7 
Vinay explained that the strategy is to go to away players in order to get less interrupts.

g6
Ananya said they are maximizing most people to talk with. If they found soul mate then talk for all the time. Low wisdom may be an aftermath of the “hello” technique. 
Amar believes that the clustering they do is the reason for low wisdom.

g9 vs g2
g9 
Samarth describes the strategy as try to go to 2 meters. If they are interrupted then try to go closer.
Ananya asked if the player 0 is doing the “hello” technique. 

Robert added that if small parties all players have time to talk with everybody.

Big party configuration.
We should not put too much friends because at the end you will get all wisdom.

g5 (took most wisdom)
They are going for the closest after end of conversation

Orestis added that if you are in big party go simple. If you are in small go for soulmate.

Vinay added that if many friends is difficult to find the soulmate. if small number of people you will find your soulmate for sure.
Amar pointed out that going 6m they lose too much space.

Is good to search for your soulmate?
Robert replied that if you do not have any friends then is the best to go for soulmate.

Lingyan asked g5 if they have a threshold for interrupt.
Jing replied that they have and also are going closer to the guy talking to.

Tingting believes that you should weigh differently friends and strangers for knowing where to go for better wisdom.

Kevin added that if there is 3 persons trying to do conversation the 5 ticks can be advantageous.
Robert believes that g5 has an advantage because of g8 as they are waiting for strangers.

Artur added that you can predict behaviour even without talking to people. This can be done with some more analytics about the players you encounter.
Manyi believed that they should try to maximize the people talking to outside how much wisdom they are getting. The answer was that they should not do this because it is not the spirit of the game.

They team should start thinking about the 5 ticks wait period to maximize conversation time.

\subsection{Monday November 30, 2015}
A conversation started today if finding your soulmate gives any advantage.
Sagar replied that talking to strangers is better than talking to your soulmate.
Seth added that talking to soulmate is good but in terms of points is the same as talking to anybody. So we should double the points we are getting from them.
Kevin believes that we should have a function asking if we have more wisdom in order not to know if it is our soulmate but to learn through the conversation.
Costa pointed out that finding your soulmate just gives another player with more time. does not give any advantage.

We took a poll about seth proposal (double how much soulmate gives)
Was voted in favor.
Costa also would like to have a response if we found the soulmate.
Sagar would like to cut the time in half for the soulmate in order to have more 
time for other players.
Professor replied that 20 minutes is minimum for talking with your soulmate.

Configuration of the experiment:
Amar suggested that we should go big 300 people with 10 friends.

Robert pointed out that they got a 35\% increase when they changed the threshold 
from 0.5 to 0.6. Because with the calculations they were getting 0.499…. which 
was not passing the simulator.
Amar added that as many teams were going for 5 clicks they also change to wait 
if they had more wisdom to gain from the people they were talking to.
Artur believes if more than 100 people in the party is not wishful to wait.
g5 player strategy as explained by sagar is that they go for the best wisdom guy at 6m radius
Also now they are using a heuristic to understand how close they are to find their soulmate.
Ananya added that the g6 fixed the problem of gathering. Also they had some problems with their soulmate not wanting to talk so they stalked. 
Sagar said they never had any such problem as they genral they are not wander around.
They were a group at the corner and Amar wanted to know if we know which group this is.
Preetam suggested that they have gain all the wisdom and went there in order not to give any more. 
Lingyan pointed out that in a such party they are able to drain the wisdom.
Seth believed that maybe it is a destination point.
Cathy said that they (g1) are probably the one that created this group. Because 
when do clustering they go to the denser area. So things can add up.
Derek suggested that if they found dense group should not go. If moderate then go.
Pathi added that hardcoded locations should be avoided as can lead to such clusters.
Alice pointed out that they do not have hardcoded location. They just went to denser cluster.
Robert said that they also tried clustering but it was too bad because of density. If you end up with a dense are you are surrounded and you cannot be productive.
Cathy believed that going greedy in big party can create such groups.
Ananya added that they are detecting if being in dense area. First by inactivity (no wisdom for many ticks) and for people are in our 6m range. If found then go to a random place and start a conversation.
As productivity is what matters Dhruv said that if too many click inactive then go for a conversation
Sagar added that only if you do not have any to talk go for 6m.
Samarth also was in favor for such technique. As it gives more space to find player.
Vinay believed that you should search first nearby and then started going to different location.
Costa added that every move should be beneficiary.
Parthi also said that not always go for random because if you are in a corner you can end up more in the corner.


A new configuration run with 100 people and 5 friends.

professor asked how the players are ranking people.
g7 said that if more than one then go for more wisdom.
Vinay said that are going for the closest.
Manyi also pointed out that the first option maybe is talking.
Artur added that if many people in the party do not go for small wisdom.
Manyi said that keeping track of players will give you an insight in order not to go for some players that may not want to talk to you.
Sagar added that they have an average mechanism to report how much wisdom is left between 2 players
Sania also believes that having an estimate of wisdom can save from unnecessary conversation.
Parthi added that you should have a good estimate by the time you broke up the conversation with each player.
Jing pointed out that some certain situations can get some more wisdom.

Professor asked how you estimate the wisdom.
Seth pointed out that if you got chased then they have some more wisdom.
Robert added that you can get from behaviour. If they are stalking you they have more. If run then they do not have more.
Amar suggested to count up what wisdom you accumulate and not down. We know the limits 10.20 ,.. so try to find how many more you have.
Sagar pointed out that breaking after one tick is not bad as you may want to come closer.Start few conversations from afar. 
Dhruv added that some are trying to go closer but are starting conversation from 2m.

Professor asked how many are going for 0.6.
Half the groups replied yes.
Kevin suggested not to go immediately to closer location. Wait to be interrupted because you may lose ticks.
Dhruv also pointed out that 0.6 is not the best as it may still go closer and lose the turn.
Sagar mentioned that you should go for a bit pivoting and not in straight line.
Samarth opinion is the one which we talked last time the converged one.
Seth points out that if you do not go direct maybe they will misinterpret your move.

What is the most important thing for next time:
Samarth says that with new reward find soulmate is something new.
Cathy believes that finding a threshold to start a conversation distance is crucial
Amar added a destructive technique. If you add all wisdom go to the corner and wait.
Dhruv also pointed out that different strategies for large/small parties can be created.
Sagar suggested that the whole purpose is to minimize bad ticks -ticks with no wisdom.
So if you wait 5 ticks you lose a big chunk of time.
\subsection{Wednesday December 2, 2015}
The conversation started with the asking for possible scenarios to run

Dhruv asked for one with no friends. As this will give better results and will have the soulmate for some teams.
Sean replied that a small party will affect best if you found or not your soulmate.
Derek said that a configuration with 50 people will be the best as efficient teams can collect all the wisdom.
Seth pointed out that if there are less people then all will find their soulmate
Sagar added that with no friends the strategy is simple.


100 people - 0 friends

Costa said as g7 had the best wisdom that they added the potential wisdom code and also started going closer.
Sagar from g5 stated that they change their code to be disruptive. They saw that with their previous code they were chasing down people to talk. So they started going disturbing people. Their new approach did not gave them better results but lower all others significantly.
Nam proposed banning also interrupting people to lower their wisdom.
Costa added that as we do not have communication we can do anything as a team.
Preetam told that we can come and go in order to do the same strategy as g5 is doing.
Robert believes that if the soulmate is close they can resume the conversation.
Also after Robert’s question g7 replied that they are not doing anything for the disruptive strategy of g5.
Amar said that they are trying to have better strategies for people with potential more wisdom.
Sean also added that they added a wisdom threshold so the soulmate will have better waiting.
Sagar pointed out that not all interruptions are result of their player. So it is still better not to do anything as it is probably a pass by.
Vinay added that you should not have all your players disruptive but have a portion of them at the beginning.
Robert also pointed that there is too much randomness so you can end up with all your soulmates to the same disruptive player.
Vishal believed is feasible to change behaviour amongst the different interactions (friend, stranger, soulmate) .

A conversation started from having 2 kind of players from a team.

Amar believes that having a universal player is best as you will have better results for many instances.
Costa wants to add the wingman instance that if someone is talking with his soulmate to interrupt the interrupter. 
Dhruv points out that having 2 strategies that are doing great is not possible. Every group should have organizers and workers. Only with the correct balance you can achieve the best.
Sagar believes that a counter strategy with the interruption is to go for a particular instance to have the best score. Then the rest will pick up some wisdom. As the disruptive will lower the rest players you will get more.

This tight conversation arrangement gave the idea of square dance proposed by professor.
Parthi pointed out that not all conversations are good as you may have more strangers with low wisdom than friends.
Vishal believes that as all teams will speak with similar numbers of people then the cumulative wisdom would similar.
Kevin proposed to have a loose square dance with empty spaces to have players in and out to have more random results.
Preetam added that with such strategy teams that are selfless will have best results as they will not preempt.

A poll took place about the square dance.
Only 1 vote from Parthi.
Parthi told that in such game all will try to get the best out of each player.
Sagar believes that if all players are coordinating then interrupting is easier.
Alice added that this incentive is like speed dating in which the main problem is that the 2 parts does not have big wisdom. So for us is not the best solution to have something like this.
Seth added that it is as a multiplayer prisoner’s dilemma as if one is not cooperating then everybody will lose.

The conversation ended there as there was not much time. Professor would like to have a small conversation about the tournament.

Saman would like to have some direct tournaments between 2 players.
Ananya proposed a much /less friends parties in order to give soulmate a favor
Dhruv requested parties where all players can gather the wisdom in order to test efficiency.
Seth would like some dense boards where conversation can be interrupted easily.
Amar idea is to see some percent of friends (0,10\%,20\%).
Sagar pointed out that if not many friends then strangers are getting more value along the soulmate.

Professor tried to initiate a conversation about what is a good strategy for dense parties.
Sania said that going to edge will give you less plane to be interfered.
Alice added that then a corner will also give you less plane.
Derek pointed out that if everybody does that then the middle will be sparse.
Sania continued saying that you can create clusters of wisdom and try to go near them.
Parthi point of view was that if too dense then no one will get much wisdom. So 
split team in half and point half to a corner/edge and rest talking. In the middle 
of the game change teams.

\subsection{Monday December 14, 2015}
The duration of the presentation is 5 minutes. George will give a signal when one
minute will be left.

Group 3:
The team presented a problem analysis. The main question was 
\begin{itemize}
\item How to select the next person to talk to?
\item How to avoid interference?
\item When do we favor the soulmate?
\item How to take advantage of friends?
\item When and where to move around the room?
\end{itemize}

The team used a cost function:
\begin{equation*}
cost = \frac{scale\_dmin*scaleOfFreeCandidate*wisdom}{scale\_distance}
\end{equation*}
where:
\begin{itemize}
\item scale\_dmin: scale down people who are currently interfered
\item scale of free candidate: give higher priority to free people
\item scale\_distance: give higher priority to people that we are the closest to 
\end{itemize}

Some times stalking the soulmate is not the correct way. If talking to soulmate 
gives is favorable talk to him if is in talking range and is the closest person
to us and our conversation succeeded at least once in the last 3 turns. Else
move closer to the soulmate. They had 2 strategies for candidate selection. One 
greedy to talk to the closest one or a backup one to move closer to a person.
Their main strategy is to move closer to a target and have a brief history in 
order to understand if he is interested. They are trying not to waste time on
uninterested players or when a region is too crowded. Their implementation 
was not that good from the results they had. But they went well in cases with
0 friends in the room.

Group 5:
The group used 2 strategies. A constructive one and with a portion of their players
a destructive one. They have a prediction about the wisdom that is dynamically 
updated. In their constructive strategy they wait for the interference to pass.
Otherwise they can move away from the interrupter or move closer to the target.
Their destructive strategy is focused on lowering the average score of all.
They found out that with 3\% of destructive players is ideal to have maximum 
interference. In their results they found that without their presence 4 groups
did worse than group 0 (default player).

Group 7:
This group's main strategy is to find your friends and acquire their wisdom. Then
go to the person that is the most far away and has the most wisdom. This is in 
order not to be interfered. And at last a jump function. This function is 
different from moving because it makes you jump in a place where it is less 
probable to be interfered. They did well and had a good average score. They were
second in wins and overall many times in the podium. Groups 2,7,9 led the parties 
to higher scores.

Group 2:
Group 2 has an overall strategy to assign some expected wisdom, continue conversation
if one is already in place, a waiting function in order to understand if the
person has more wisdom to share when interrupted. Also they blacklist players
and having a function to move towards to best players. The expected wisdom is
calculated by $expected wisdom = \frac{Total wisdom}{s+1}$ where s is the strangers
and $Total Wisdom = 10*s + 400$. 400 is the soulmate wisdom and 10 is the average
wisdom amongst strangers. The wisdom-dependent waiting function is calculated:
\begin{equation*}
T = 2 + \frac{P_{remaining-wisdom}}{20}
\end{equation*}
The team also sorts stranger in order to keep track to people that have talked 
before in order not to waste time to someone that have not much to share. They 
also have a scoring function that makes not lose time to strangers with not much
wisdom.
\begin{equation*}
\begin{aligned}
expected\_rate\_of\_wisdom\_gain\_by\_exhausting\_the\_target’s\_wisdom &=\\
 \frac{remaining\_wisdom}{time\_to\_intiate\_conversation + \frac{remaining\_wisdom}{(1- probability\_of\_interruption)* widom\_rate}} \\
\end{aligned}
\end{equation*}
\begin{equation*}
\begin{aligned}
remaining\_wisdom &= \frac{\#\_of\_people\_we\_want\_to\_talk\_to}{\#\_of\_people\_in\_range\_of\_us}*\#\_of\_people\_in\_range\_of\_target \\
probability\_of\_interruption &= 1 - (1-\frac{3}{18\pi})^n \\
\end{aligned}
\end{equation*}
where n is the number of people within distance 3 to conversation point. Their 
blacklist strategy is if a player leaves after or 6 or 60 seconds (1 or 10 wisdom 
points) as it means it will not start again a conversation. An exception is made
if the player moved to us in a straight line. They did great won 841 games and
a player of them got the highest score at 697 games. But they did not do well
against themselves as too many unnecessary iterations to speak to each other.

Group 1:
Group 1 has split the problem on sparse and dense board strategies. Also on how 
to gain more wisdom on situation with less or plenty strangers. They are initialize
the average wisdom for each stranger as: $E(stranger's wisdom) = \frac{(0+10+20)/3 * (\#\_of\_strngers-1)+400}{\#\_of\_strangers}$\\
Another part of the created strategy is the waiting decision and how to find next
conversation.
For the latter they have implemented a KMeamns algorithm in order to analyze
where sparse or dense groups are in the board. From their experiments saw that 
random jumps to region gave better result from informed ones. Their wait decision
strategy is to start with a constant N but they changed into some more intelligent.
They always wait N turns but it is changed along with how much wisdom someone
gave $N = floor(\frac{wisdom}{6})$. Also if someone ignored their request, they
black listed him as probably they had less wisdom to give than to gain. They have
a scoring function in order to sort the possible strangers that are around 
$expexcted\_wisdom = current\_wisdom-6*(\#\_of\_players\_with\_less\_0.5\_distance)$.
If the player is not in walking distance then another 6 is subtracted. Also if 
the player is our soulmate double the number that we calculated. They have as a 
threshold to start conversation the 0.25 distance. They move halfway every tick 
in order to achieve this. If no applicable candidate is around then move 6 units
towards the middle. Their results are better in sparse configuration with low number
of friends. They understood that spent too much time in where to jump with KMeans
but in the end random jumps gave better results.

Group 8:
The group implemented a maximum likelihood estimator for unseen strangers. Their
main strategy is a greedy one that finds the person with the highest wisdom in 6m
range. Then they are moving near the person at 0.505 to initiate conversation.
The expected value of an unseen stranger than is not your soulmate is as follows
\begin{equation*}
E[W[s]] = \frac{1}{n_r}(n_{0r}E_{us}[0] + n_{10r}E_{us}[10] + n_{20r}E_{us}[20])
\end{equation*}
where $n_r$ is the number of remaining unseen strangers, $n_{0r},n_{10r},n_{20r}$
are the number of unseen 0-point, 10-point, and 20-point strangers respectively
$E_{us}[0]=0,E_{us}[10] = 2/3 *10,E_{us}[20] = 1/3*20+1/3*10$
Their strategy is most effective when large fraction of groups are not executing 
similar strategies. When large number of players are aggressive then the board
is becoming destructive as are resulting in clustering effects and interference.
They have also implemented some adaptive behavior on the party size, the waiting
time for a stranger and how all these are affected on how much time we have left.
But they had a bug on the soulmate so their results are not the best possible.
They are 5th on average score but 2nd on a mini tournament they held with the bug
fixed.

Group 9:
The main issues for this group are how to handle interrupted conversations, how
to move closer to someone you want to talk without losing time and also how to
prioritize targets. For choosing the next target is just greedy on the closest
friend that is not talking and has some wisdom. The move around is also greedy 
as they move to the visible friend with the most wisdom. In order to interact with
the player they are not going extremely close in order not to cancel the move
with the same but reverse from the other player. If soulmate is visible then has
the best priority even if is further away. In case of interference wait 3 cycles
before moving closer. If you are stuck for 6 cycles move randomly. Their results
show that is the group with the higher percentage of meeting the soulmate. In 
general they have good distribution on their rankings. They have found that getting
more people in the party will lower the gained wisdom because of interference.

Group 4:




\end{document}

