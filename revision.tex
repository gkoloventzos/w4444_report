\section{Course revision}
\subsection{Wednesday December 9, 2015}
The class started with a conversation about the last project if they students liked 
it. Costa was the first to answer and said that it was a tough one because you
could change some parameters and everything will change. Dhruv said that in contrast
with the previous projects the GUI was too messy so it was hard to debug. Sean
added that small scale test was the best in order to understand what is going on.
Manyi also commented that you debug with small and then proceed to larder boards.
Vinay's opinion is that the GUI was fine and it gave you the clustering view you 
wanted in order to debug clustering problems. Sagar believes that having some more
ticks to view the changes would be better.

For the wrap up Professor started with the question how many times you have gone
through the notes. By raising hands the majority of the students said once or twice.
In the following question of why Preetam answered that went to see the configurations
for the tournament. Dhruv read the notes for some ideas and cite in a presentation.

The Professor asked about what have you thought this course would be. Robert's 
answer was that in the end the solutions that come out from the conversations
were more complex than we should had. Sagar added that he expected to read more
literature and apply algorithms to problems even for the first time. Professor 
replied that no one forbid you from doing so. That was part of the idea behind 
of mixing groups and projects in order to motivate you search behind the problem.
Artur said that was hard to explore other solutions because the time each had to
create something for next class was too little. Costa added that for the second
project a class discussing the physics and maths behind the project should have 
been held. Professor opinion was that in order to not have this you got the document
with all of them (you can see this document in appendix \ref{appendix}). Parthi
has the opinion that more algorithmic strategies should have been implemented. 
Like every team should implement different algorithm as at the end we will compete.
That was the complain of Vishal. Most of the teams were trying to beat each other
and not solve the problem. Amar added that if someone presents a nice player on 
Monday then everybody and you will try for that approach for next deliverable,
Then it is too much to rewrite all the code on a different approach. Robert 
continues that someone finds a local maximum and then everybody stays around that
instead of pushing for something else. Preetam believes that going from Monday
to Wednesday deliverable is a scavenger hunt for best code. Not many people have
the same opinion with Preetam. Cathy added that this was a problem for the project
2. A best solution was found and then everybody converge to this. So the same
strategy was done in the next project. Derek asked if project 3 had same solution
from more than one team. Professor answered that the teams converged to shape.
Artur also said again about the problem of teams focusing on beating the others
instead of thinking how to implement something better. Sagar believes that students
were focusing on a promising solution. More solutions should be provided by the
TAs and/or the professor. Seth added that different teams should for different 
approaches. Costa believes that the TAs should provide with a best player that
teams should try to beat in order not to converge. Professor then asked which 
project was your favorite. Most student voted for the first project (19). Second
came the third (13). The forth project came third (9) and the second came last (5).
Professor asked which project should be played next year and how should it change.

Project 1:
Preetam said that a board with all teams will be nice as the sweep strategies will
claim less rats. Artur believes that not permanently claiming rats will add more
hustle and less stealing. Robert thought is that music should have volume that 
will be connected with the speed of the piper. Vinay said that pipers should also
have a health bar in order to have more control. Also adding more things to be 
lured, e.g. like children. Alice added that would be nice to have rats with size.

Project 2:
Seth pleaded not the be the first project as it will a huge obstacle for the students.
Robert believes that gravity assist will be a nice addition. Amar added that
moving to a 3 dimension model will make it more interesting. Dhruv thinks that 
collisions should be elastic to have some nice pool effects. 

Project 3:
Amar's thoughts are that more players should play and do let the pieces to be picked
randomly. Manyi believes that it would be a nice addition to add some cost per
piece. The player will start with a quota and it will have to play in order not
to run out. Robert added to this thought that each piece should add some quota 
and then spend it. Dhruv suggested that same pieces should be allowed in order
to test algorithms. Professors reactions is it would be interesting if we pick 
the pieces for them? Nam believes that it would be as now your strategy will have
to be more generic. Rhea added that the picking itself has strategy on its own.

Project 4:
Costa's addition would be to give the wisdom at the end of the conversation and 
not in every tick. Parthi's thought is to give more visible space but less travel
distance.

Professor asked what would like to change or not in the next course. Artur said
that having this 2 classes per week is nice as it creates a schedule. Robert
believes that a bit more of scaffolding would be nice like functions to change.
Amar also added that giving help will be better as the distance function in 
planet building project. Professor answered that they do not want to guide to a
specific solution. Also adding a repository for the code would be better in order
to distribute the code better. Parthi then asked professor what you want the 
students to get. Professor answered that applying algorithms and things you have
learned is a main goal. Also to develop creativity, your teamwork and how cs can
be a nice game. Samarth then said that the presentation at the end was also good
because it gives an overall of your strategy and the rest of the teams. Vinay also
thing that adding a real world problem would be a nice addition. Tingting added
that as the simulator had visualization was really helpful.
